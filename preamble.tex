%%%%%%%%%%%%%%%%%%%%%%%%%%%%%%%%%%%%%%%%%%%%%%%%%%%%%%%%%%%%%%%%%%%%%%
%% Packages

\usepackage{scrhack}

\usepackage[ngerman,main=american]{babel}

\usepackage{calc}

\usepackage{printlen}

\usepackage{nicefrac}
\usepackage{xfrac}
\usepackage[super]{nth} %\nth{1}, \nth{2}, \nth{3}, \nth{4}

% SI units
\usepackage{eurosym}
\usepackage[binary-units=true]{siunitx}
\sisetup{
    % quotient-mode = fraction,
    per-mode = fraction,
    fraction-function=\sfrac,
    % product-units = single,
    % output-product={\,},
    unit-color = UNITCOLOR,
    % round-mode=places,
    % round-precision=2,
    % exponent-product = \cdot,
    % output-decimal-marker = {.},
    % fraction=sfrac,
    group-separator = {\,},
    group-minimum-digits = 3,
}
% \sisetup{}
\DeclareSIUnit \var { \si{\volt}\si{\ampere}r } %volt-ampere reactive - idle
% power
\DeclareSIUnit \EUR { \text{\euro} } %euro as unit
\DeclareSIUnit\year{a}

\usepackage{xargs}       % Use more than one optional parameter in a new commands

\usepackage{complexity}

\usepackage{pifont} % for \ding{55} -> cross

% Figures
\usepackage{graphicx}
\input{insbox}
\usepackage{wrapfig}
\usepackage{cutwin}

% \usepackage[redeflists]{IEEEtrantools}

% Table
\usepackage{tabularx}
\usepackage{multirow}
\usepackage{booktabs}
\def\sharedaffiliation{%
  \end{tabular}
  \begin{tabular}{c}
}
\usepackage{colortbl}
\usepackage{pdflscape}
% \usepackage{lscape}

% Quotes
\usepackage{csquotes}

\usepackage{timetravel}
\setcounter{equation}{70}
\setCounters{chapter,equation,section,figure}%algocfproc,theorem,algocf



\makeatletter%https://tex.stackexchange.com/questions/265841/small-overleftrightarrow
    \newcommand{\overleftrightsmallarrow}{\mathpalette{\overarrowsmall@\leftrightarrowfill@}}
    \newcommand{\overrightsmallarrow}{\mathpalette{\overarrowsmall@\rightarrowfill@}}
    \newcommand{\overleftsmallarrow}{\mathpalette{\overarrowsmall@\leftarrowfill@}}
% 
    \newcommand{\overarrowsmall@}[3]{%
    \vbox{%
      \ialign{%
        ##\crcr
        #1{\smaller@style{#2}}\crcr
        \noalign{\nointerlineskip}%
        $\m@th\hfil#2#3\hfil$\crcr
      }%
    }%
    }
    \def\smaller@style#1{%
    \ifx#1\displaystyle\scriptstyle\else
      \ifx#1\textstyle\scriptstyle\else
        \scriptscriptstyle
      \fi
    \fi
    }
\makeatother

%% New environment for research question
% \DeclareFloatingEnvironment{researchquestion}[Research Question][List of
% research questions]
% \crefformat{enumi}{#2Reseacrh Questions #1#3}
% \renewenvironment{researchquestion}
%   {\begin{enumerate}}
%   {\end{enumerate}}

%%%%%%%%%%%%%%%%%%%%%%%%%%%%%%%%%%%%%%%%%%%%%%%%%%%%%%%%%%%%%%%%%%%%%%
%% Indentation

\newlength{\myparindent}
\newlength{\myparskip}
\setlength{\myparindent}{1em}
\setlength{\myparskip}{0em}

\setlength{\parindent}{\myparindent}
\setlength{\parskip}{\myparskip}
\setlength{\parskip}{0pt plus1pt minus0pt}


%%%%%%%%%%%%%%%%%%%%%%%%%%%%%%%%%%%%%%%%%%%%%%%%%%%%%%%%%%%%%%%%%%%%%%
%% Colors

% \usepackage[dvipsnames]{xcolor}
\usepackage[table,x11names,dvipsnames,table]{xcolor}
\usepackage[most]{tcolorbox}

% Thesis, KIT, and power grid colors
\usepackage{transparent}

% Thesis colors
\definecolor{thesisblue}         {rgb}{0.337, 0.592, 0.773}
\definecolor{thesisblue-dark}    {rgb}{0.212, 0.471, 0.655}
\definecolor{thesisblue-light}   {rgb}{0.490, 0.710, 0.867}
\definecolor{thesisblue-vlight}  {rgb}{0.871, 0.937, 0.988}

\definecolor{thesisred}          {rgb}{0.776, 0.357, 0.396}
\definecolor{thesisred-dark}     {rgb}{0.663, 0.235, 0.271}
\definecolor{thesisred-light}    {rgb}{0.867, 0.502, 0.533}
\definecolor{thesisred-vlight}   {rgb}{0.984, 0.871, 0.878}

\definecolor{thesisgreen}        {rgb}{0.337, 0.765, 0.235}
\definecolor{thesisgreen-dark}   {rgb}{0.267, 0.612, 0.184}
\definecolor{thesisgreen-light}  {rgb}{0.443, 0.871, 0.341}
\definecolor{thesisgreen-vlight} {rgb}{0.710, 0.984, 0.643}

\definecolor{thesisyellow}       {rgb}{0.808, 0.659, 0.263}
\definecolor{thesisyellow-dark}  {rgb}{0.667, 0.529, 0.180}
\definecolor{thesisyellow-light} {rgb}{0.898, 0.765, 0.412}
\definecolor{thesisyellow-vlight}{rgb}{0.992, 0.933, 0.776}

\definecolor{black}              {rgb}{0,0,0}
\definecolor{black70}            {rgb}{0.30,0.30,0.30}
\definecolor{black50}            {rgb}{0.50,0.50,0.50}
\definecolor{black30}            {rgb}{0.70,0.70,0.70}
\definecolor{black15}            {rgb}{0.85,0.85,0.85}
\definecolor{black7}             {rgb}{0.95,0.95,0.95}

% Set major colors
\colorlet{primarycolor-dark}      {thesisblue-dark}
\colorlet{primarycolor}           {thesisblue}
\colorlet{primarycolor-light}     {thesisblue-light}
\colorlet{primarycolor-vlight}    {thesisblue-vlight}
\colorlet{secondarycolor-dark}    {thesisred-dark}
\colorlet{secondarycolor}         {thesisred}
\colorlet{secondarycolor-light}   {thesisred-light}
\colorlet{secondarycolor-vlight}  {thesisred-vlight}

\colorlet{stroke1}                {primarycolor-dark}
\colorlet{stroke2}                {secondarycolor-dark}

% Set document colors
\colorlet{captionlabel}{black}
\colorlet{footerpagenr}{black}
\colorlet{footerchapter}{stroke1}
\colorlet{footerchaptername}{black}
\colorlet{footersection}{stroke1}
\colorlet{footersectionname}{black}
\colorlet{chapternumber}{stroke1}
\colorlet{sectiontitlenumber}{black}
\colorlet{sectiontitle}{black}

% ACM already defined colors
\definecolor[named]{ACMBlue}{cmyk}{1,0.1,0,0.1}
\definecolor[named]{ACMYellow}{cmyk}{0,0.16,1,0}
\definecolor[named]{ACMOrange}{cmyk}{0,0.42,1,0.01}
\definecolor[named]{ACMRed}{cmyk}{0,0.90,0.86,0}
\definecolor[named]{ACMLightBlue}{cmyk}{0.49,0.01,0,0}
\definecolor[named]{ACMGreen}{cmyk}{0.20,0,1,0.19}
\definecolor[named]{ACMPurple}{cmyk}{0.55,1,0,0.15}
\definecolor[named]{ACMDarkBlue}{cmyk}{1,0.58,0,0.21}

%  
% \definecolor{orange}{HTML}{FF7F00}
% \definecolor{orange}{rgb}{1,0.5,0}
% \definecolor{orange}{RGB}{255,127,0}
% \definecolor{figChar}{cmyk}{0,0.90,0.86,0}

% KIT Colors
% KIT green
\definecolor{KITgreen}{rgb}{0 0.588 0.509}
\definecolor{KITgreen70}{rgb}{0.3 0.711 0.656}
\definecolor{KITgreen50}{rgb}{0.5 0.794 0.754}
\definecolor{KITgreen30}{rgb}{0.7 0.876 0.852}
\definecolor{KITgreen15}{rgb}{0.85 0.938 0.926}
% KIT palegreen
\definecolor{KITpalegreen}{rgb}{0.509,0.745,0.235}
\definecolor{KITpalegreen70}{rgb}{0.656,0.821,0.464}
\definecolor{KITpalegreen50}{rgb}{0.754,0.872,0.617}
\definecolor{KITpalegreen30}{rgb}{0.852,0.923,0.77}
\definecolor{KITpalegreen15}{rgb}{0.926,0.961,0.885}
% KIT yellow
\definecolor{KITyellow}{rgb}{0.98,0.901,0.078}
\definecolor{KITyellow70}{rgb}{0.986,0.931,0.354}
\definecolor{KITyellow50}{rgb}{0.99,0.95,0.539}
\definecolor{KITyellow30}{rgb}{0.994,0.97,0.723}
\definecolor{KITyellow15}{rgb}{0.997,0.985,0.861}
% KIT orange
\definecolor{KITorange}{rgb}{0.862,0.627,0.117}
\definecolor{KITorange70}{rgb}{0.903,0.739,0.382}
\definecolor{KITorange50}{rgb}{0.931,0.813,0.558}
\definecolor{KITorange30}{rgb}{0.958,0.888,0.735}
\definecolor{KITorange15}{rgb}{0.979,0.944,0.867}
% KIT brown
\definecolor{KITbrown}{rgb}{0.627,0.509,0.196}
\definecolor{KITbrown70}{rgb}{0.739,0.656,0.437}
\definecolor{KITbrown50}{rgb}{0.813,0.754,0.598}
\definecolor{KITbrown30}{rgb}{0.888,0.852,0.758}
\definecolor{KITbrown15}{rgb}{0.944,0.926,0.879}
% KIT red
\definecolor{KITred}{rgb}{0.627,0.117,0.156}
\definecolor{KITred70}{rgb}{0.739,0.382,0.409}
\definecolor{KITred50}{rgb}{0.813,0.558,0.578}
\definecolor{KITred30}{rgb}{0.888,0.735,0.747}
\definecolor{KITred15}{rgb}{0.944,0.867,0.873}
% KIT lilac
\definecolor{KITlilac}{rgb}{0.627,0,0.47}
\definecolor{KITlilac70}{rgb}{0.739,0.3,0.629}
\definecolor{KITlilac50}{rgb}{0.813,0.5,0.735}
\definecolor{KITlilac30}{rgb}{0.888,0.7,0.841}
\definecolor{KITlilac15}{rgb}{0.944,0.85,0.92}
% KIT blue
\definecolor{KITblue}{rgb}{0.274,0.392,0.666}
\definecolor{KITblue70}{rgb}{0.492,0.574,0.766}
\definecolor{KITblue50}{rgb}{0.637,0.696,0.833}
\definecolor{KITblue30}{rgb}{0.782,0.817,0.9}
\definecolor{KITblue15}{rgb}{0.891,0.908,0.95}
% KIT seablue
\definecolor{KITseablue}{rgb}{0.196,0.313,0.549}
\definecolor{KITseablue70}{rgb}{0.437,0.519,0.684}
\definecolor{KITseablue50}{rgb}{0.598,0.656,0.774}
\definecolor{KITseablue30}{rgb}{0.758,0.794,0.864}
\definecolor{KITseablue15}{rgb}{0.879,0.897,0.932}
% KIT cyanblue
\definecolor{KITcyanblue}{rgb}{0.313,0.666,0.901}
\definecolor{KITcyanblue70}{rgb}{0.519,0.766,0.931}
\definecolor{KITcyanblue50}{rgb}{0.656,0.833,0.95}
\definecolor{KITcyanblue30}{rgb}{0.794,0.9,0.97}
\definecolor{KITcyanblue15}{rgb}{0.897,0.95,0.985}
% KIT black
\definecolor{KITblack}{rgb}{0,0,0}
\definecolor{KITblack70}{rgb}{0.3,0.3,0.3}
\definecolor{KITblack50}{rgb}{0.5,0.5,0.5}
\definecolor{KITblack30}{rgb}{0.7,0.7,0.7}
\definecolor{KITblack15}{rgb}{0.85,0.85,0.85}

% Power Grid specific colors
\definecolor{FEASIBLE}{rgb}{0.739,0.3,0.629}
\definecolor{PHYSICAL}{rgb}{0.437,0.519,0.684}
\definecolor{KVL}{rgb}{0.437,0.519,0.684}
\definecolor{KCL}{rgb}{0.739,0.3,0.629}
% Problems
\definecolor{MTSF}{rgb}{0.509,0.745,0.235}
\definecolor{MPF}{rgb}{0.437,0.519,0.684}
\definecolor{MF}{rgb}{0.739,0.3,0.629}
\definecolor{OTS}{rgb}{0,0,0}
\definecolor{ROP}{rgb}{0,0,0}
\definecolor{ROP}{rgb}{0,0,0}
\definecolor{MFF}{rgb}{0,0,0}
\definecolor{OPF}{rgb}{0,0,0}
% 
\definecolor{HELMHOLTZblue}{rgb}{0.043,0.362,0.618}
\definecolor{HELMHOLTZgreen}{rgb}{0.549,0.7,0.19}
\definecolor{CHANGES}{rgb}{0.997,0.985,0.861}
% Elements
\definecolor{GENERATOR}{rgb}{0.437,0.519,0.684}
\definecolor{CONSUMER}{rgb}{0.739,0.3,0.629}
\definecolor{SUSCEPTANCE}{rgb}{0.862,0.627,0.117}
\definecolor{SWITCHE}{rgb}{0,0.588,0.509}
\definecolor{THETA}{rgb}{0.313,0.666,0.901}
\definecolor{CAPACITY}{rgb}{0.5,0.5,0.5}
\definecolor{DTP}{rgb}{0.627,0.117,0.156}

\colorlet{Table-Line-Marker}  {KITyellow15}
\colorlet{DUALGRAPH}          {THETA}
\colorlet{PRIMALGRAPH}        {KITblack50}
\colorlet{KVLCONFLICT}        {KITred70}
\colorlet{SUBSTATION}         {KITbrown70}
\colorlet{VOLTAGE}            {KITblue70}
\colorlet{CURRENT}            {KITred70}
\colorlet{REALPOWER}          {KITcyanblue70}
\colorlet{REACTIVEPOWER}      {KITcyanblue30}
\colorlet{COMPLEXPOWER}       {KITgreen70}
\colorlet{POWERANGLE}         {KITpalegreen70}
\colorlet{TIMESTAMP}          {KITblack70}
\colorlet{UNITCOLOR}          {KITblack70}
\colorlet{ColorTableRule}     {KITblack70}
\colorlet{ColorKclConflictMarker}       {KITred70}
\colorlet{ColorSusceptanceScalingMarker}{SUSCEPTANCE}
\colorlet{ColorFACTSedge}     {KITred70}

% Wind farm cabling
\colorlet{ColorTransmissionCable}{KITseablue}
\colorlet{ColorSubstation}       {KITbrown}
\colorlet{ColorCollectionPoint}  {ColorSubstation}
\colorlet{ColorTransportCableSmall}     {KITblack50}
\colorlet{ColorTransportCableMedium}    {KITblack70}
\colorlet{ColorTransportCableLarge}     {KITblack}
\colorlet{ColorCircuitProblem}          {KITgreen}
\colorlet{ColorSubstationProblem}       {KITorange}
\colorlet{ColorFullWindFarm}            {KITred}



%%%%%%%%%%%%%%%%%%%%%%%%%%%%%%%%%%%%%%%%%%%%%%%%%%%%%%%%%%%%%%%%%%%%%%
%% Link
\newcommandx{\screen}[1]{\textcolor{black}{#1}}%
\newcommandx{\screentextcolor}[2]{\textcolor{#1}{#2}}%
\colorlet{HiLi}{KITorange15}

%%%%%%%%%%%%%%%%%%%%%%%%%%%%%%%%%%%%%%%%%%%%%%%%%%%%%%%%%%%%%%%%%%%%%%
%% Highlight text background

\usepackage{soul}
\DeclareRobustCommand{\hlhili}[1]{{\sethlcolor{HiLi}\hl{#1}}}
\DeclareRobustCommand{\myhl}[2]{{\sethlcolor{#1}\hl{#2}}}


%%%%%%%%%%%%%%%%%%%%%%%%%%%%%%%%%%%%%%%%%%%%%%%%%%%%%%%%%%%%%%%%%%%%%%
%% ...

\newlength{\tmpa}
\newlength{\tmpb}
\newlength{\tmpc}


%%%%%%%%%%%%%%%%%%%%%%%%%%%%%%%%%%%%%%%%%%%%%%%%%%%%%%%%%%%%%%%%%%%%%%
%% Size and position of the text body

\newlength{\mybodywidth}
\newlength{\mybodyheight}
\newlength{\myoutermargin}
\newlength{\mytopmargin}
\setlength{\mybodywidth}{132.5mm}
\setlength{\mybodyheight}{195mm}
\setlength{\myoutermargin}{51.8mm}%+5mm}
\setlength{\mytopmargin}{38mm}

\newlength{\myinnermargin}
\newlength{\mybottommargin}
\setlength{\myinnermargin}{\paperwidth-\mybodywidth-\myoutermargin}
\setlength{\mybottommargin}{\paperheight-\mybodyheight-\mytopmargin}


%%%%%%%%%%%%%%%%%%%%%%%%%%%%%%%%%%%%%%%%%%%%%%%%%%%%%%%%%%%%%%%%%%%%%%
%% Position of the cut marks

\newlength{\cutmarktop}
\newlength{\cutmarkbottom}
\newlength{\cutmarkinner}
\newlength{\cutmarkouter}

\setlength{\cutmarktop}{7mm}
\setlength{\cutmarkbottom}{7mm}
\setlength{\cutmarkinner}{0mm}
\setlength{\cutmarkouter}{7mm}


%%%%%%%%%%%%%%%%%%%%%%%%%%%%%%%%%%%%%%%%%%%%%%%%%%%%%%%%%%%%%%%%%%%%%%
%% Position of the head and foot

\newcommand{\goldenratio}{1.618}

\newlength{\myheadsep} % distance from header to body
\setlength{\myheadsep}{(\mytopmargin-\cutmarktop) -
  (\mytopmargin-\cutmarktop) / \real{\goldenratio} - 1ex}

\newlength{\myfootskip} % distance from body to footer
\setlength{\myfootskip}{(\mybottommargin-\cutmarkbottom) /
  \real{\goldenratio} - 1ex}


%%%%%%%%%%%%%%%%%%%%%%%%%%%%%%%%%%%%%%%%%%%%%%%%%%%%%%%%%%%%%%%%%%%%%%
%% Margin position and width

\newlength{\mymargininnersep} % distance between margin and body
\newlength{\mymarginoutersep} % distance between margin and paper edge
\newlength{\mymarginwidth} % width of the margin
\setlength{\mymargininnersep}{7mm}
\setlength{\mymarginoutersep}{12mm}
\setlength{\mymarginwidth}{\myoutermargin - \mymargininnersep -
  \mymarginoutersep - \cutmarkouter}


%%%%%%%%%%%%%%%%%%%%%%%%%%%%%%%%%%%%%%%%%%%%%%%%%%%%%%%%%%%%%%%%%%%%%%
%% Applying the geometry

\usepackage[
paper=a4paper,
width=\mybodywidth,
height=\mybodyheight,
outer=\myoutermargin,
top=\mytopmargin,
headsep=\myheadsep,
footskip=\myfootskip,
marginparsep=\mymargininnersep,
marginparwidth=\mymarginwidth
% showframe
]{geometry}

%%%%%%%%%%%%%%%%%%%%%%%%%%%%%%%%%%%%%%%%%%%%%%%%%%%%%%%%%%%%%%%%%%%%%%
%% Comments and TODOs

% for new line use "\newline"
\usepackage[colorinlistoftodos,prependcaption,textsize=tiny]{todonotes}
% 
\newcommandx{\franzi}[2][1=]{\todo
[linecolor=red,backgroundcolor=red!25,bordercolor=red,#1]{{\textbf{
\underline{Franzi:}}}\newline#2}\xspace}

%%%%%%%%%%%%%%%%%%%%%%%%%%%%%%%%%%%%%%%%%%%%%%%%%%%%%%%%%%%%%%%%%%%%%%
%% Printing crop marks

\usepackage{tikz}
\setlength{\tmpa}{1mm}
\newcommand{\cropmarkeven}{%
  \begin{tikzpicture}[overlay,remember picture]%
    \node [inner sep=0,outer sep=0, anchor=north west] at (current
    page.north west) {%
      \begin{tikzpicture}[overlay,remember picture]%
        \draw (\cutmarkouter,0mm) -- (\cutmarkouter,-\cutmarktop+\tmpa);%
        \draw (0mm,-\cutmarktop) -- (\cutmarkouter-\tmpa,-\cutmarktop);%
      \end{tikzpicture}%
    };%
    \node [inner sep=0,outer sep=0, anchor=north east] at (current
    page.north east) {%
      \begin{tikzpicture}[overlay,remember picture]%
        \draw (0mm,-\cutmarktop) -- (-\cutmarkouter+\tmpa,-\cutmarktop);%
      \end{tikzpicture}%
    };%
    \node [inner sep=0,outer sep=0, anchor=south west] at (current
    page.south west) {%
      \begin{tikzpicture}[overlay,remember picture]%
        \draw (\cutmarkouter,0mm) -- (\cutmarkouter,\cutmarktop-\tmpa);%
        \draw (0mm,\cutmarktop) -- (\cutmarkouter-\tmpa,\cutmarktop);%
      \end{tikzpicture}%
    };%
    \node [inner sep=0,outer sep=0, anchor=south east] at (current
    page.south east) {%
      \begin{tikzpicture}[overlay,remember picture]%
        \draw (0mm,\cutmarktop) -- (-\cutmarkouter+\tmpa,\cutmarktop);%
      \end{tikzpicture}%
    };%
  \end{tikzpicture}
}
\newcommand{\cropmarkodd}{%
  \begin{tikzpicture}[overlay,remember picture]%
    \node [inner sep=0,outer sep=0, anchor=north west] at (current
    page.north west) {%
      \begin{tikzpicture}[overlay,remember picture]%
        \draw (0mm,-\cutmarktop) -- (\cutmarkouter-\tmpa,-\cutmarktop);%
      \end{tikzpicture}%
    };%
    \node [inner sep=0,outer sep=0, anchor=north east] at (current
    page.north east) {%
      \begin{tikzpicture}[overlay,remember picture]%
        \draw (-\cutmarkouter,0mm) -- (-\cutmarkouter,-\cutmarktop+\tmpa);%
        \draw (0mm,-\cutmarktop) -- (-\cutmarkouter+\tmpa,-\cutmarktop);%
      \end{tikzpicture}%
    };%
    \node [inner sep=0,outer sep=0, anchor=south west] at (current
    page.south west) {%
      \begin{tikzpicture}[overlay,remember picture]%
        \draw (0mm,\cutmarktop) -- (\cutmarkouter-\tmpa,\cutmarktop);%
      \end{tikzpicture}%
    };%
    \node [inner sep=0,outer sep=0, anchor=south east] at (current
    page.south east) {%
      \begin{tikzpicture}[overlay,remember picture]%
        \draw (-\cutmarkouter,0mm) -- (-\cutmarkouter,\cutmarktop-\tmpa);%
        \draw (0mm,\cutmarktop) -- (-\cutmarkouter+\tmpa,\cutmarktop);%
      \end{tikzpicture}%
    };%
  \end{tikzpicture}
}

\usepackage{everypage}
\usepackage[strict]{changepage}

\newcommand{\showcropmarks}{%
  \AddEverypageHook{%
    \checkoddpage%
    \ifoddpage%
    \cropmarkodd%
    \else%
    % \cropmarkeven%
    \fi%
  }}


%%%%%%%%%%%%%%%%%%%%%%%%%%%%%%%%%%%%%%%%%%%%%%%%%%%%%%%%%%%%%%%%%%%%%%
%% Header and footer format

\usepackage{fancyhdr}
\pagestyle{fancy} 

\fancypagestyle{plain}{%
  \fancyhf{} % clear header and footer

  % no header/footer line
  \renewcommand{\headrulewidth}{0pt}
  \renewcommand{\footrulewidth}{0pt}

  % set distances
  \fancyheadoffset[EL]{\myoutermargin}
  \fancyheadoffset[OR]{\myoutermargin}
  \fancyfootoffset[EL]{\myoutermargin}
  \fancyfootoffset[OR]{\myoutermargin}

  % page numbers
  \fancyfoot[EL]{%
    {\color{stroke1}\rule{\myoutermargin-\mymargininnersep}{0.5mm}}\\
    \begin{minipage}[b]{\myoutermargin-\mymargininnersep}%
      \raggedleft\color{footerpagenr}\textbf{\thepage}%
    \end{minipage}%
  }
  \fancyfoot[OR]{%
    {\color{stroke1}\rule{\myoutermargin-\mymargininnersep}{0.5mm}}\\
    \begin{minipage}[b]{\myoutermargin-\mymargininnersep}%
      \raggedright\color{footerpagenr}\textbf{\thepage}%
    \end{minipage}%
  }
}

\fancypagestyle{content}{%
  % use default page numbering
  \pagestyle{plain}

  % define header
  \renewcommand{\chaptermark}[1]{%
    \markboth{%
      \small\textsf{\color{footerchaptername}##1}%
    }{}%
  }
  \renewcommand{\sectionmark}[1]{%
    \markright{%
      \small\textsf{\color{footersectionname}##1}%
    }%
  } 
  \fancyhead[EL]{%
    \begin{minipage}[b]{\myoutermargin-\mymargininnersep}%
      \small%
      \raggedleft\textsf{\textbf{\color{footerchapter}\chaptername\ \thechapter}}
    \end{minipage}%
    \hspace{\mymargininnersep}\leftmark%
  }
  \fancyhead[OR]{%
    \rightmark\hspace{\mymargininnersep}%
    \begin{minipage}[b]{\myoutermargin-\mymargininnersep}%
      \small%
      \raggedright\textsf{\textbf{\color{footersection}Section\ \thesection}}
    \end{minipage}%
  }
}


%%%%%%%%%%%%%%%%%%%%%%%%%%%%%%%%%%%%%%%%%%%%%%%%%%%%%%%%%%%%%%%%%%%%%%
%% Caption format

\usepackage{caption}
\captionsetup{
  font=small,
  labelfont={bf,sf,color=captionlabel},
  format=plain,
  singlelinecheck=off
  %tableposition=top
}

\usepackage{floatrow}
\floatsetup[table]{capposition=top}
% \usepackage{colortbl}
% \usepackage[table]{xcolor}
% \usepackage{multirow}


%%%%%%%%%%%%%%%%%%%%%%%%%%%%%%%%%%%%%%%%%%%%%%%%%%%%%%%%%%%%%%%%%%%%%%
%% part format
% \RequirePackage{silence}
% \WarningFilter{scrreprt}{Usage of package `titlesec'}
% \WarningFilter{scrreprt}{Activating an ugly workaround}

\usepackage[noindentafter,explicit]{titlesec}

\assignpagestyle{\part}{empty}
\titleformat{\part}
[display]
{\Huge\sffamily}%
{}%
{-2\baselineskip}%
{%
  \setlength{\tmpa}{0.618\paperwidth}%
  \setlength{\tmpb}{0.382\paperheight}%
  \begin{tikzpicture}[overlay,remember picture]%
    \node [inner sep=0,outer sep=0, anchor=north] at (current
    page.north west) {%
      \begin{tikzpicture}[overlay,remember picture]%
        \draw[color=stroke1, line width=0.7mm] (\tmpa,0) -- (\tmpa,-\tmpb);%
      \end{tikzpicture}%
    };%
    \node (align) [align=right, below=\tmpb-2ex, inner sep=0, outer
    sep=0, anchor=north west] at (current page.north west){%
      \hspace{\tmpa}\hspace{0.5em}\partname\ \thepart\\[1ex]%
      \color{stroke1}#1%
    };%
  \end{tikzpicture}%
}%
[]


%%%%%%%%%%%%%%%%%%%%%%%%%%%%%%%%%%%%%%%%%%%%%%%%%%%%%%%%%%%%%%%%%%%%%%
%% Chapter format
% 
\let\oldchapter\chapter
\newcommand\Chapter[3]{%
% \renewcommand\chapter[3]{%
  \def\tempSubtitle{#2}
  \def\tempFooter{#3}
  % 
  \ifx\tempSubtitle\empty
    \def\tempTocEntry{#1}
    \def\tempHeadEntry{#1}
    \ifx\tempFooter\empty
      \def\tempHeader{\sffamily\fontsize{19.5}{19.5}\selectfont#1}
    \else
      \def\tempHeader{\sffamily\fontsize{19.5}{19.5}\selectfont#1#3}
    \fi
  \else
    \def\tempTocEntry{#1\hfil\hbox{}\protect\linebreak{\normalfont\itshape#2}}%
    % \def\tempTocEntry{#1\protect\linebreak{\normalfont\itshape#2}}%
    \def\tempHeadEntry{#2}
    \ifx\tempFooter\empty
      \def\tempHeader{\sffamily\fontsize{19.5}{19.5}\selectfont#1\\[0.0ex]\Large\itshape#2}
    \else
      \def\tempHeader{\sffamily\fontsize{19.5}{19.5}\selectfont#1#3\\[0.0ex]\Large\itshape#2}
    \fi
  \fi
  % 
  \oldchapter[%
    tocentry={\tempTocEntry},
    head={\tempHeadEntry}
  ]%
  {
    \ifx\tempSubtitle\empty
      \sffamily\fontsize{19.5}{19.5}\selectfont#1\hspace{-2mm}#3
    \else
      \sffamily\fontsize{19.5}{19.5}\selectfont#1\hspace{-2mm}#3
      \\[0.0ex]\Large\itshape#2
    \fi
  }%
}

\usepackage{xpatch}
\makeatletter
\providecommand*\@chaptersubtitle{}
\FamilyStringKey[.section]{KOMAarg}{chaptersubtitle}{\@chaptersubtitle}
\xpretocmd{\chapter}{\xdef\@chaptersubtitle{}}{}{\PatchFailed}
\xapptocmd\chapterlinesformat{%
  \ifstr{\@chaptersubtitle}{}{}
    {%
      \parbox[t]{\linewidth}
        {\raggedchaptersubtitle\usekomafont{chaptersubtitle}\@chaptersubtitle}%
      \par\nobreak
    }%
  }
  {}{\PatchFailed}
\xapptocmd\chapterlineswithprefixformat{%
  \ifstr{\@chaptersubtitle}{}{}
    {%
      \parbox[t]{\linewidth}
        {\raggedchaptersubtitle\usekomafont{chaptersubtitle}\@chaptersubtitle}%
      \par\nobreak
    }%
  }
  {}{\PatchFailed}
\makeatother
\newcommand\raggedchaptersubtitle{\raggedchapter}
\newkomafont{chaptersubtitle}{\normalsize}



% \assignpagestyle{\chapter}{plain}
\titlespacing{\chapter}{0mm}{0mm}{0mm}
\titleformat{\chapter}[display]%
{\sffamily\color{stroke1}\huge}% format
{}% label
{0pt}% sep
{%
  \begin{tikzpicture}[overlay,remember picture]%
    \node [right=\myinnermargin,below=\mytopmargin, inner sep=0, outer
    sep=0, anchor=north west] (numbernode) at (current page.north west){%
      \hspace{\myinnermargin}%
      \sffamily\fontsize{60}{60}\selectfont%
      \color{chapternumber}%
      \raggedchaptersubtitle%
      \thechapter%
    };%
    \node [inner sep=0,outer sep=0, anchor=north west] at (numbernode.south west) {%
      \begin{tikzpicture}[overlay,remember picture]%
        \draw[color=stroke1, line width=0.7mm] (\myinnermargin,-1ex) -- (\paperwidth,-1ex);%
      \end{tikzpicture}%
    };%
    \node (align) [align=right, right=\myinnermargin+\mybodywidth, inner sep=0, outer sep=0,
    anchor=east] at (numbernode.west) {%
      #1%
    };%
  \end{tikzpicture}%
}% before-code
[\bigskip]% after-code


\titleformat{name=\chapter,numberless}
[display]%
{\sffamily\color{stroke1}\huge}% format
{}% label
{0pt}% sep
{%
  \begin{tikzpicture}[overlay,remember picture]%
    \node [right=\myinnermargin,below=\mytopmargin, inner sep=0, outer
    sep=0, anchor=north west] (numbernode) at (current page.north west){%
      \hspace{\myinnermargin}%
      \sffamily\fontsize{60}{60}\selectfont%
      \color{white}%
      \thechapter%
    };%
    \node [inner sep=0,outer sep=0, anchor=north west] at (numbernode.south west) {%
      \begin{tikzpicture}[overlay,remember picture]%
        \draw[color=stroke1, line width=0.7mm] (\myinnermargin,-1ex) -- (\paperwidth,-1ex);%
      \end{tikzpicture}%
    };%
    \node (align) [align=left, right=\myinnermargin, inner sep=0, outer sep=0,
    anchor=south west] at (numbernode.south west) {%
      #1%
    };%
  \end{tikzpicture}%
}% before-code
[\bigskip]% after-code


%%%%%%%%%%%%%%%%%%%%%%%%%%%%%%%%%%%%%%%%%%%%%%%%%%%%%%%%%%%%%%%%%%%%%%
%% Section format

% \titleformat{\section}%
% [block]%
% {\Large\bfseries\sffamily}%
% {\color{sectiontitlenumber}\thesection\hspace{0.5em}}%
% {0pt}%
% {\color{sectiontitle}\raggedright}%
% []


%%%%%%%%%%%%%%%%%%%%%%%%%%%%%%%%%%%%%%%%%%%%%%%%%%%%%%%%%%%%%%%%%%%%%%
%% Subsection format

% \titleformat{\subsection}%
% [block]%
% {\large\bfseries\sffamily}%
% {\color{sectiontitlenumber}\thesubsection\hspace{0.5em}}%
% {0pt}%
% {\color{sectiontitle}\raggedright}%
% []


%%%%%%%%%%%%%%%%%%%%%%%%%%%%%%%%%%%%%%%%%%%%%%%%%%%%%%%%%%%%%%%%%%%%%%
%% Paragraph format

\titleformat{\paragraph}%
[runin]%
{\bfseries\sffamily}%
{}%
{0pt}%
{#1.}%
[]


%%%%%%%%%%%%%%%%%%%%%%%%%%%%%%%%%%%%%%%%%%%%%%%%%%%%%%%%%%%%%%%%%%%%%%
%% Biblatex and CV packages

\usepackage[utf8]{inputenc}

\usepackage{currvita}
\usepackage{csquotes}
\usepackage[
style=alphabetic,
% style=trad-alpha,
%style=numeric,
autocite=plain,
maxnames=100,
backend=biber,
natbib=true,
defernumbers,
safeinputenc,
backref=true,
backrefstyle=none,
sortcites,
maxcitenames=2,
% abbreviate=true 
% citetracker=true,
% sortcites=true, 
% sorting=ynt, 
% pagetracker=true,
% backref=true,
% minnames=1,
% maxbibnames=10,
% minbibnames=7,
% uniquelist=true,
% uniquename=allfull,
]{biblatex}
% \autocite, \textcite,\parencite
\patchcmd{\bibsetup}{\interlinepenalty=5000}{\interlinepenalty=10000}{}{}

% Set up back references.
\DefineBibliographyStrings{english}{%
  backrefpage = {cited on page}, % originally "cited on page"
  backrefpages = {cited on pages}, % originally "cited on pages"
}

\DeclareFieldFormat{pagerefformat}{#1}
% \DeclareFieldFormat{pagerefformat}{\small\color{stroke1}\mkbibemph{{#1}}}

\renewbibmacro*{pageref}{%
  \addperiod%
  \iflistundef{pageref}%
    {}%
    {\newline\footnotesize\printtext[pagerefformat]{%
       \ifnumgreater{\value{pageref}}{1}%
         {\bibstring{backrefpages}\ppspace}%
     {\bibstring{backrefpage}\ppspace}%
       \printlist[pageref][-\value{listtotal}]{pageref}\addperiod}}}

% Set up bibliography layout.
\DeclareNumChars*{:}
\renewcommand{\intitlepunct}{\addspace}
\renewcommand{\multicitedelim}{\addcomma\space}

\renewbibmacro{in:}{%
  \ifentrytype{article}{}{\printtext{\bibstring{in}\intitlepunct}}}

\renewbibmacro*{volume+number+eid}{
  \printfield{volume}
  \iffieldundef{number}{}{\addcolon}%
  \printfield{number}
  \setunit*{\addcomma\space}
  \printfield{eid}
}

\renewbibmacro*{editor+others}{%
  \ifboolexpr{
    test \ifuseeditor
    and
    not test {\ifnameundef{editor}}
  }
    {\printnames{editor}%
     \setunit{\space}%
     \usebibmacro{editor+othersstrg}%
     \clearname{editor}}
    {}}

\addbibresource{bibliography/article.bib}
\addbibresource{bibliography/book.bib}
\addbibresource{bibliography/thesis.bib}
\addbibresource{bibliography/internet.bib}
% \addbibresource{bibliography/list-of-publications.bib}

% \AtEveryCitekey{\ifciteseen{}{\defcounter{maxnames}{99}}}

\DeclareFieldFormat[article]{title}{\textbf{\color{stroke1}#1}}
\DeclareFieldFormat[inproceedings]{title}{\textbf{\color{stroke1}#1}}
\DeclareFieldFormat[thesis]{title}{\textbf{\color{stroke1}#1}}
\DeclareFieldFormat[book]{title}{\textbf{\color{stroke1}#1}}
\DeclareFieldFormat[unpublished]{title}{\textbf{\color{stroke1}#1}}
\DeclareFieldFormat[report]{title}{\textbf{\color{stroke1}#1}}
\DeclareFieldFormat[inbook]{chapter}{\textbf{\color{stroke1}#1}}
\DeclareFieldFormat[inbook]{title}{\textbf{\color{stroke1}#1}}
\DeclareFieldFormat{editortype}{\mkbibparens{#1}}
\DeclareFieldFormat[misc]{title}{\textbf{\color{stroke1}#1}}
\DeclareFieldFormat[manual]{title}{\textbf{\color{stroke1}#1}}

\DefineBibliographyStrings{english}{%
  andothers = {\textsc{et\,al\adddot}},
  editor  = {editor},
  editors = {editors},
  page    = {page},
  pages   = {pages},
  volume  = {volume},
}

\newtoggle{authorend}
\togglefalse{authorend}

\DeclareBibliographyDriver{article}{%
  \usebibmacro{bibindex}%
  \usebibmacro{begentry}%
  \iftoggle{authorend}{}{\usebibmacro{author/translator+others}}%
  \setunit{\labelnamepunct}\newblock
  \usebibmacro{title}%
  \newunit
  \printlist{language}%
  \newunit\newblock
  \usebibmacro{byauthor}%
  \newunit\newblock
  \usebibmacro{bytranslator+others}%
  \newunit\newblock
  \printfield{version}%
  \newunit\newblock
  \usebibmacro{journal}
  \setunit{\addnbspace}
  \usebibmacro{volume+number+eid}
  \newunit
  \usebibmacro{byeditor+others}%
  \newunit
  \usebibmacro{note+pages}%
  \setunit{\addcomma\space}
  \printfield{year}
  \newunit\newblock
  \iftoggle{bbx:isbn}
  {\printfield{issn}}
  {}%
  \newunit\newblock
  \usebibmacro{doi+eprint+url}%
  \newunit\newblock
  \usebibmacro{addendum+pubstate}%
  \setunit{\bibpagerefpunct}\newblock
  \usebibmacro{pageref}%
  \newunit\newblock
  \iftoggle{bbx:related}
  {\usebibmacro{related:init}%
    \usebibmacro{related}}
  {}%
  \usebibmacro{finentry}%
  \iftoggle{authorend}{\usebibmacro{author/translator+others}}{}%
}

\DeclareBibliographyDriver{inbook}{%
  \usebibmacro{bibindex}%
  \usebibmacro{begentry}%
  \iftoggle{authorend}{}{\usebibmacro{author/translator+others}}%
  \setunit{\labelnamepunct}\newblock
  \printfield{title}%
  \newunit
  \printlist{language}%
  \newunit\newblock
  \usebibmacro{byauthor}%
  \newunit\newblock
  \usebibmacro{in:}%
  \usebibmacro{maintitle+booktitle}%
  \newunit\newblock
  \usebibmacro{byeditor+others}%
  \newunit\newblock
  \printfield{edition}%
  \iffieldundef{volume}
  {\setunit{\addcomma\space}}
  {  \newunit\printfield{volume}%
    \printfield{part}\space\bibstring{ofseries}\space\printfield{series}\setunit{\addcomma\space}}%
  \printfield{pages}
  \newunit\newblock
  \printfield{note}%
  \newunit\newblock
  \usebibmacro{publisher+location+date}%
  \newunit\newblock
  \iftoggle{bbx:isbn}
  {\printfield{isbn}}
  {}%
  \newunit\newblock
  \usebibmacro{doi+eprint+url}%
  \newunit\newblock
  \usebibmacro{addendum+pubstate}%
  \setunit{\bibpagerefpunct}\newblock
  \usebibmacro{pageref}%
  \newunit\newblock
  \iftoggle{bbx:related}
  {\usebibmacro{related:init}%
    \usebibmacro{related}}
  {}%
  \usebibmacro{finentry}%
  \iftoggle{authorend}{\usebibmacro{author/translator+others}}{}%
}

\DeclareBibliographyDriver{book}{%
  \usebibmacro{bibindex}%
  \usebibmacro{begentry}%
  \usebibmacro{author/editor+others/translator+others}%
  \setunit{\labelnamepunct}\newblock
  \usebibmacro{maintitle+title}%
  \newunit
  \printlist{language}%
  \newunit\newblock
  \usebibmacro{byauthor}%
  \newunit\newblock
  \usebibmacro{byeditor+others}%
  \newunit\newblock
  \printfield{edition}%
  \newunit
  \iffieldundef{volume}
  {\setunit{\addcomma\space}}
  {  \newunit\printfield{volume}%
    \printfield{part}\space\bibstring{ofseries}\space\printfield{series}\setunit{\addcomma\space}}%
  %
  \newunit\newblock
  \printfield{note}%
  \newunit\newblock
  \usebibmacro{publisher+location+date}%
  \newunit
  \iflistundef{location}%
  {\setunit*{\addcomma\space}}%
  {\setunit*{\addcomma\space}}%
  \printlist{location}%
  {}
  \newunit
  \usebibmacro{chapter+pages}%
  \newunit
  \printfield{pagetotal}%
  \newunit\newblock
  \iftoggle{bbx:isbn}%
  {\printfield{isbn}}%
  {}
  \newunit\newblock
  \usebibmacro{doi+eprint+url}%
  \newunit\newblock
  \usebibmacro{addendum+pubstate}%
  \newunit\newblock
  \usebibmacro{pageref}%
  \usebibmacro{finentry}
}

\DeclareBibliographyDriver{inproceedings}{%
  \usebibmacro{bibindex}%
  \usebibmacro{begentry}%
  \iftoggle{authorend}{}{\usebibmacro{author/translator+others}}%
  \setunit{\labelnamepunct}\newblock
  \usebibmacro{title}%
  \newunit
  \printlist{language}%
  \newunit\newblock
  \usebibmacro{byauthor}%
  \newunit\newblock
  \usebibmacro{in:}%
  \usebibmacro{maintitle+booktitle}%
  \newunit\newblock
  \usebibmacro{event+venue+date}%
  \newunit\newblock
  \usebibmacro{byeditor+others}%
  \newunit\newblock
  \iffieldundef{volume}
  {}
  {\printfield{volume}%
    \printfield{part}\space\bibstring{ofseries}}
  \printfield{series}%
  \newunit\newblock
  \usebibmacro{chapter+pages}%
  \iflistundef{publisher}%
  {\setunit{\addcomma\space}}%
  {\newunit\newblock}%
  \usebibmacro{publisher+location+date}%
  \newunit\newblock
  \iftoggle{bbx:isbn}
  {\printfield{isbn}}
  {}%
  \newunit\newblock
  \usebibmacro{doi+eprint+url}%
  \newunit\newblock
  \usebibmacro{addendum+pubstate}%
  \setunit{\bibpagerefpunct}\newblock
  \usebibmacro{pageref}%
  \newunit\newblock
  \iftoggle{bbx:related}
  {\usebibmacro{related:init}%
    \usebibmacro{related}}
  {}%
  \usebibmacro{finentry}%
  \iftoggle{authorend}{\usebibmacro{author/translator+others}}{}%
}

\DeclareBibliographyDriver{thesis}{%
  \usebibmacro{bibindex}%
  \usebibmacro{begentry}%
  \iftoggle{authorend}{}{\usebibmacro{author}}%
  \setunit{\labelnamepunct}\newblock
  \usebibmacro{title}%
  \newunit
  \printlist{language}%
  \newunit\newblock
  \usebibmacro{byauthor}%
  \newunit\newblock
  \printfield{note}%
  \newunit\newblock
  \printfield{type}%
  \newunit
  \usebibmacro{institution+location+date}%
  \newunit\newblock
  \usebibmacro{chapter+pages}%
  \newunit
  \printfield{pagetotal}%
  \newunit\newblock
  \iftoggle{bbx:isbn}
  {\printfield{isbn}}
  {}%
  \newunit\newblock
  \usebibmacro{doi+eprint+url}%
  \newunit\newblock
  \usebibmacro{addendum+pubstate}%
  \setunit{\bibpagerefpunct}\newblock
  \usebibmacro{pageref}%
  \newunit\newblock
  \iftoggle{bbx:related}
  {\usebibmacro{related:init}%
    \usebibmacro{related}}
  {}%
  \usebibmacro{finentry}%
  \iftoggle{authorend}{\usebibmacro{author}}{}%
}

\DeclareBibliographyDriver{misc}{%
  \usebibmacro{bibindex}%
  \usebibmacro{begentry}%
  \iftoggle{authorend}{}{\usebibmacro{author/translator+others}}%
  \setunit{\labelnamepunct}\newblock
  \usebibmacro{title}%
  \newunit
  \printlist{language}%
  \newunit\newblock
  \usebibmacro{byauthor}%
  \newunit\newblock
  \printfield{note}%
  \newunit\newblock
  \printfield{type}%
  \space
  \printfield{number}%
  \newunit\newblock
  \usebibmacro{institution+location+date}%
  \newunit
  \usebibmacro{doi+eprint+url}%
  \newunit\newblock
  \usebibmacro{addendum+pubstate}%
  \setunit{\bibpagerefpunct}\newblock
  \usebibmacro{pageref}%
  \newunit\newblock
  \iftoggle{bbx:related}
  {\usebibmacro{related:init}%
    \usebibmacro{related}}
  {}%
  \usebibmacro{finentry}%
  \iftoggle{authorend}{\usebibmacro{author/translator+others}}{}%
}

% Copied from numeric.cbx to imitate numerical citations.
\providebool{bbx:subentry}
\newbibmacro*{citenum}{%Note: the original macro was called "cite". I did not redefine "cite", but instead defined a new macro "citenum", because the author-year citations use the "cite" macro too. "\renewbibmacro*{cite}" would have caused all the author-year citations to become numeric too.
  \printtext[bibhyperref]{%If you ever want to use hyperref
    \printfield{prefixnumber}%
    \printfield{labelnumber}%
    \ifbool{bbx:subentry}
    {\printfield{entrysetcount}}
    {}}}

% Copied from numeric.cbx to define a new numeric citation command for @online entries.
\DeclareCiteCommand{\conline}[\mkbibbrackets]
{\usebibmacro{prenote}}
{\usebibmacro{citeindex}%
  \usebibmacro{citenum}}%Note: this was originally "cite" but I changed it to "citenum" to avoid clashes with the author-year style.
{\multicitedelim}
{\usebibmacro{postnote}}


%%%%%%%%%%%%%%%%%%%%%%%%%%%%%%%%%%%%%%%%%%%%%%%%%%%%%%%%%%%%%%%%%%%%%%
%% Other packages 

\usepackage[T1]{fontenc}

\usepackage[final]{microtype}

\usepackage{amsmath}

\usepackage{amssymb}

\usepackage{amsthm}

\usepackage{thmtools}

\usepackage[
hidelinks,
% ocgcolorlinks=true,
% colorlinks=false,
%linkcolor=stroke1,
%urlcolor=stroke1,
%citecolor=stroke2,
linkbordercolor=primarycolor-vlight,
urlbordercolor=primarycolor-vlight,
citebordercolor=secondarycolor-vlight,
pdftex,
pdfauthor={Franziska Wegner},
pdftitle={Combinatorial Problems in Energy Networks}
]{hyperref}

% \usepackage[ocgcolorlinks]{ocgx2}
% \usepackage{ocgx2}
\usepackage{accents}
\usepackage{graphicx}

\usepackage{paralist}

\usepackage{booktabs}

\usepackage{multirow}

\usepackage{rotating}

\usepackage{xspace}

\usepackage[absolute,overlay]{textpos}

\usepackage[ttscale=.85]{libertine}

\usepackage[libertine,cmintegrals,cmbraces,vvarbb]{newtxmath}

\usepackage{timetravel}
\setCounters{theorem}

\usepackage{accents}

\usepackage{mathtools}
\DeclarePairedDelimiter{\ceil}{\lceil}{\rceil}

\usepackage{subcaption}
\captionsetup[subfigure]{justification=centering}

\usepackage[innercaption]{sidecap}
\columnsep=\mymargininnersep

\usepackage{pgfplots}
\usepackage{filecontents}
\pgfplotsset{compat=1.9}
\usepgfplotslibrary{external} 
\tikzexternalize[prefix=tikz/]
\tikzexternaldisable

\usepackage{environ}
\makeatletter
\newsavebox{\measure@tikzpicture}
\NewEnviron{scaletikzpicturetowidth}[1]{%
  \def\tikz@width{#1}%
  \def\tikzscale{1}\begin{lrbox}{\measure@tikzpicture}%
  \BODY
  \end{lrbox}%
  \pgfmathparse{#1/\wd\measure@tikzpicture}%
  \edef\tikzscale{\pgfmathresult}%
  \BODY
}
\makeatother


%%%%%%%%%%%%%%%%%%%%%%%%%%%%%%%%%%%%%%%%%%%%%%%%%%%%%%%%%%%%%%%%%%%%%%
%% Algorithm stuff

% \usepackage[algochapter, algo2e, figure, ruled, vlined, linesnumbered]
% {algorithm2e}
% \usepackage{listings}
\usepackage[titlenumbered, ruled, vlined, linesnumbered,resetcount]
{algorithm2e}%, dotocloa
\setCounters{algocf}%algocfproc,
% \setcounter{algorithm}{1}
\SetKwComment{Comment}{\color{KITblack30}$\triangleright$\ }{}
\SetCommentSty{small}
\SetKw{kwnot}{not}
\SetKw{kwand}{and}
\SetKw{kwdominates}{dominates}
\SetKwFunction{deleteMin}{delMin}
\SetKwFunction{deleteFirst}{popFirst}
\SetKwFunction{pushBack}{pushBack}
\SetKwFunction{update}{update}
\SetKwFunction{myinsert}{insert}
\SetKwFunction{getPaths}{getPaths}
\SetKwFunction{deleteDominatedLabels}{deleteDominatedLabels}
\SetKwFunction{isReachable}{isReachable}
\SetKwFunction{algodfs}{dfs}
\SetKwFunction{algoplanarembedding}{planarEmbeddingOf}
\SetKwFunction{algoGridEmbeddingOfPlanarGraph}{gridEmbeddingOf}
\SetKwFunction{algoContractGridGraphToEdge}{contractGridGraphToEdge}
\SetKwFunction{algoDecontractEdgeToGridGraph}{decontractEdgeToGridGraph}
\SetKwFunction{algoconstructDualGraphOf}{constructDualGraphOf}
\SetKwFunction{algoBipolarSubgraphOf}{bipolarSubgraphOf}
\SetKwFunction{algoDualGraphOf}{dualGraphOf}
\SetKwFunction{algoMinCostFlow}{minCostFlow}
\SetKwFunction{resolveConflict}{resolveKclConflict}
\SetKwFunction{sweepLineFlowAlgorithm}{sweepLineFlowAlgorithm}
\SetKwFunction{shortestPathGraph}{shortestPathGraph}
\SetKwFunction{algoRescalePowerFlow}{rescalePowerFlow}
% \DontPrintSemicolon
\SetKwFunction{topologicSort}{topologicSort}
\SetKwFunction{computeColors}{computeColors}
\SetKw{kwfrom}{from}
\SetKw{kwto}{to}
\SetKw{kwcontinue}{continue}



\newcommand{\assign}{\longleftarrow}

\SetAlFnt{\small}
\SetAlTitleFnt{\small}
\newcommand{\commentfont}[1]{{\color{black50}#1}}
\newcommand{\kwfont}[1]{\texttt{\color{stroke1}#1}}
\newcommand{\funcfont}[1]{\texttt{#1}}
\newcommand{\datafont}[1]{\texttt{#1}}

\SetCommentSty{commentfont}
\SetKwSty{kwfont}
\SetFuncSty{funcfont}
\SetDataSty{datafont}

\SetKwFunction{queueIsEmpty}{isEmpty}%
\SetKwFunction{queueIsNotEmpty}{isNotEmpty}%
\SetKwFunction{queueMinElement}{minElement}%
\SetKwFunction{queueDeleteMin}{deleteMin}%
\SetKwFunction{queueDeleteMax}{deleteMax}%
\SetKwFunction{queueInsert}{insert}%
\SetKwFunction{queueUpdate}{update}%
\SetKwFunction{setInsert}{insert}%
\SetKwFunction{setDeleteDominatedLabels}{deleteLabelsDominatedBy}%
\SetKw{KwNotDominates}{does not dominate}%

\makeatletter
\newcommand{\setalgotoprulecolor}[1]{\colorlet{toprulecolor}{#1}}
\let\old@algocf@pre@ruled\@algocf@pre@ruled
\renewcommand{\@algocf@pre@ruled}{\textcolor{toprulecolor}{\old@algocf@pre@ruled}}

\newcommand{\setalgobotrulecolor}[1]{\colorlet{bottomrulecolor}{#1}}
\let\old@algocf@post@ruled\@algocf@post@ruled
\renewcommand{\@algocf@post@ruled}{\textcolor{bottomrulecolor}{\old@algocf@post@ruled}}

\renewcommand{\algocf@caption@ruled}{\box\algocf@capbox\vspace{-3mm}
\textcolor{bottomrulecolor}{\old@algocf@post@ruled}}%
\makeatother

\setalgotoprulecolor{stroke1}
\setalgobotrulecolor{stroke1}

%%%%%%%%%%%%%%%%%%%%%%%%%%%%%%%%%%%%%%%%%%%%%%%%%%%%%%%%%%%%%%%%%%%%%%
%% Problem Environment

\makeatletter
% \renewcommand\arraystretch{1.5}
\newenvironment{problem}[2][]{%
  \def\problem@arg{#1}%
  \def\problem@framed{framed}%
  \def\problem@lined{lined}%
  \def\problem@doublelined{doublelined}%
  \ifx\problem@arg\@empty%
    \def\problem@hline{}%
  \else%
    \ifx\problem@arg\problem@doublelined%
      \def\problem@hline{\hline\hline}%
    \else%
      \def\problem@hline{\hline}%
      \def\problem@hrule{\hrule}%
    \fi%
  \fi%
  \ifx\problem@arg\problem@framed%
    \def\problem@table{\centering\tabularx{1\columnwidth}{|>
{\itshape}rX|c}}%
    \def\problem@title{\multicolumn{2}{|l|}{%
        \raisebox{-\fboxsep}{\textsc{#2}}%
      }}%
  \else
    \def\problem@table{\tabularx{1\columnwidth}{>{\bfseries}rXc}}%
    \def\problem@title{\multicolumn{2}{l}{%
        \raisebox{-\fboxsep}{\textsc{\large #2}}%
      }}%
  \fi%
  \bigskip\par\noindent%
  \renewcommand{\arraystretch}{1.2}%
    \problem@table%
      \problem@hline%
      \problem@title\\[2\fboxsep]\hspace*{0.5cm}%%
}{%
      % \\\problem@hline%
    \endtabularx%
    \problem@hrule
  \medskip\par%
}
\makeatother
%%%%%%%%%%%%%%%%%%%%%%%%%%%%%%%%%%%%%%%%%%%%%%%%%%%%%%%%%%%%%%%%%%%%%%
%% Diff Environment
\makeatletter
\providecommand*{\diff}%
{%
  \@ifnextchar^{\DIfF}{\DIfF^{}}
}%
\def\DIfF^#1{%
  \mathop{\mathrm{\mathstrut d}}%
  \nolimits^{#1}\gobblespace%
}
\def\gobblespace{%
  \futurelet\diffarg\opspace%
}
\def\opspace{%
  \let\DiffSpace\!%
  \ifx\diffarg(%
    \let\DiffSpace\relax
  \else
    \ifx\diffarg[%
      \let\DiffSpace\relax
    \else
      \ifx\diffarg\{%
        \let\DiffSpace\relax
      \fi\fi\fi\DiffSpace%
}
\makeatother
%%%%%%%%%%%%%%%%%%%%%%%%%%%%%%%%%%%%%%%%%%%%%%%%%%%%%%%%%%%%%%%%%%%%%%
%% Properties Environment
% \makeatletter
% \newenvironment{property}[1][]{%
% \textbf{Properties.}
% \begin{enumerate}
% }{%
% \end{enumerate}
% }%
% \makeatother


%%%%%%%%%%%%%%%%%%%%%%%%%%%%%%%%%%%%%%%%%%%%%%%%%%%%%%%%%%%%%%%%%%%%%%
%% Theorems

\theoremstyle{plain} %
\newtheorem{theorem}{Theorem}[chapter] %
\newtheorem{lemma}[theorem]{Lemma} %
\newtheorem{corollary}[theorem]{Corollary} %
\newtheorem{proposition}[theorem]{Proposition} %
\newtheorem{definition}[theorem]{Definition}
\newtheorem{observation}[theorem]{Observation}
\newtheorem{hypothesis}[theorem]{Hypothesis}
\newtheorem{question}[theorem]{Question}
\newtheorem{conjecture}[theorem]{Conjecture}

% \newtheoremstyle{mytheoremstyle}  % name
%     {0.25cm}                      % Space above
%     {0.25cm}                      % Space below
%     {}                            % Body font - \itshape
%     {}                            % Indent amount
%     {\bf}                         % Theorem head font - \scshape
%     {.}                           % Punctuation after theorem head
%     {.5em}                        % Space after theorem head
%     {}  % Theorem head spec (can be left empty, meaning ‘normal’)
% \theoremstyle{mytheoremstyle}
\newtheorem{reductionrule}[theorem]{Reduction Rule}
% \newtheorem{algorithm}[algocf]{Algorithm} 

%%%%%%%%%%%%%%%%%%%%%%%%%%%%%%%%%%%%%%%%%%%%%%%%%%%%%%%%%%%%%%%%%%%%%%
%% Clever referencing

\usepackage[nameinlink,capitalize]{cleveref}
\usepackage{crossreftools}%crthyperCref, \crthypercref, \crtcref, \crtCref
\newcommand{\crefrangeconjunction}{\screen{--}}
\creflabelformat{equation}{#2#1#3}

% Labeltypes of \cref
\Crefname{algorithm}{Algorithm}{Algorithms}%
\Crefname{equation}{Equation}{Equations}%
\Crefname{reductionrule}{Reduction Rule}{Reduction Rules}%
\Crefname{figure}{Figure}{Figures}%
% \Crefname{figure*}{Figure}{Figures}%
% \Crefname{plot}{Plot}{Plots}%
% \Crefname{subfigure}{Figure}{Figures}%
\Crefname{tabular}{Table}{Tables}%
\Crefname{chapter}{Chapter}{Chapters}%
\Crefname{section}{Section}{Sections}%
\Crefname{appendix}{Appendix}{Appendices}%
\Crefname{enumi}{Item}{Items}%
\Crefname{footnote}{Footnote}{Footnotes}%
\Crefname{theorem}{Theorem}{Theorems}%
\Crefname{lemma}{Lemma}{Lemmas}%
\Crefname{Lemma}{Lemma}{Lemmas}%
\Crefname{corollary}{Corollary}{Corollaries}%
\Crefname{observation}{Observation}{Observations}%
\Crefname{Observation}{Observation}{Observations}%
\Crefname{proposition}{Proposition}{Propositions}%
\Crefname{definition}{Definition}{Definitions}%
\Crefname{result}{Result}{Results}%
\Crefname{example}{Example}{Examples}%
\Crefname{remark}{Remark}{Remarks}%
\Crefname{note}{Note}{Notes}%
\Crefname{algorithm}{Algorithm}{Algorithm}%
\Crefname{line}{Line}{Lines}%
\Crefname{part}{Part}{Parts}%
\Crefname{hypothesis}{Hypothesis}{Hypotheses}%
\Crefname{question}{Question}{Questions}%
\Crefname{conjecture}{Conjecture}{Conjectures}%
% \Crefname{properties}{Property}{Properties}%
% 
% Use, e.g. Lemma x instead of Theorem x, though they use the same counter
\let\oldReductionrule\reductionrule
\renewcommand{\reductionrule}{%
  \crefalias{theorem}{reductionrule}% Theorem counter now looks like Reduction Rule
  \oldReductionrule
}
\let\oldLemma\lemma
\renewcommand{\lemma}{%
  \crefalias{theorem}{lemma}% Theorem counter now looks like Lemma
  \oldLemma
}
\let\oldObservation\observation
\renewcommand{\observation}{%
  \crefalias{theorem}{observation}% Theorem counter now looks like Observation
  \oldObservation
}
\let\oldCorollary\corollary
\renewcommand{\corollary}{%
  \crefalias{theorem}{corollary}% Theorem counter now looks like Corollary
  \oldCorollary
}
\let\oldDefinition\definition
\renewcommand{\definition}{% 
  \crefalias{theorem}{definition}% Theorem counter now looks like Definition
  \oldDefinition
}
\let\oldHypothesis\hypothesis
\renewcommand{\hypothesis}{% 
  \crefalias{theorem}{hypothesis}% Theorem counter now looks like Hypothesis
  \oldHypothesis
}
\let\oldQuestion\question
\renewcommand{\question}{% 
  \crefalias{theorem}{question}% Theorem counter now looks like Question
  \oldQuestion
}
\let\oldConjecture\conjecture
\renewcommand{\conjecture}{% 
  \crefalias{theorem}{conjecture}% Theorem counter now looks like Conjecture
  \oldConjecture
}
% \let\oldProperty\proeprty
% \renewcommand{\property}{% 
%   \crefalias{theorem}{property}% Theorem counter now looks like Property
%   \oldProperty
% }


%%%%%%%%%%%%%%%%%%%%%%%%%%%%%%%%%%%%%%%%%%%%%%%%%%%%%%%%%%%%%%%%%%%%%%
%% Acronym stuff
%% For update run: makeglossaries master

\usepackage[nopostdot,
            automake,
            % stylemods=longragged, % load glossary-longragged.sty and glossaries-extra-stylemods.sty
            style=super,
            % style=altlongragged4col,% set the glossary style
            toc
            % acronym
            ]{glossaries-extra}%,nonumberlist

% \setabbreviationstyle[acronym]{short}
\DeclareRobustCommand{\glossfirstformat}[1]{\textit{#1}}
\renewcommand*{\glsdisplayfirst}[4]{\glossfirstformat{#1#4}}

\setlength{\glspagelistwidth}{0.1\textwidth}%
\setlength{\glsdescwidth}{0.6\textwidth}%
% \renewcommand*{\glsgroupheading}[1]{\textbf{\glsgetgrouptitle{##1}\bigskip}\\}%
% \renewcommand*{\glossarypostamble}{\onecolumn}
% \newcommand*{\Agroupname}{}
% \newcommand*{\Ggroupname}{Greek letters}
\usepackage{tabu}
\newglossarystyle{mystyle}{%
  \renewenvironment{theglossary}{%
    \begin{longtabu} to \linewidth {
      X[0.25,L,b]
      X[0.40,L,b]
      X[3.5]% X[3.5,L]
      X[0.10,L]
      X[1.00,L]
    }
  }%
   % {\begin{longtabu} to \linewidth {p{0.05\linewidth}
   % p{0.05\linewidth}p{0.6\linewidth}p{0.1\linewidth}p{0.1\linewidth}}}%
   {\end{longtabu}}%% Set the table’s header: title row
  %\renewcommand*{\glossaryheader}{%
  % \bfseries Term & \bfseries Description & 
  % \bfseries Units & \bfseries Page List
  % \\\endhead}%
  % No table header:
  \renewcommand*{\glossaryheader}{}%
  % No heading between groups:
   % \renewcommand*{\glsgroupheading}[1]{\textbf{\glsgetgrouptitle{##1}}\\}%
  % Main (level 0) entries displayed in a row optionally numbered:
   \renewcommand*{\glossaryentryfield}[5]{%
      \glstarget{##1}{##2}% Name
      &
      & ##3% Description
      & \glsentryuseri{##1}% Units
      & ##5% Page list
      \\\vspace{2mm}% end of row
   }%
  % Similarly for sub-entries (no sub-entry numbers):
  \renewcommand*{\glossarysubentryfield}[6]{%
      % ignoring first argument (sub-level)
      &
      \glstarget{##2}{##3}% Name
      & ##4% Description
      & \glsentryuseri{##2}% Units
      & ##6% Page list
      \\% end of row
   }%
  % Nothing between groups:
  \renewcommand*{\glsgroupskip}{\indexspace}%
}

\newglossarystyle{unitsStyle}{%
  \renewenvironment{theglossary}{%
    \begin{longtabu} to \linewidth {
      X[0.15,L,b]
      X[0.40,L,b]
      X[4.50]
      % X[0.25,L]
      % X[0.75,L]
    }
  }%
   % {\begin{longtabu} to \linewidth {p{0.05\linewidth}
   % p{0.05\linewidth}p{0.6\linewidth}p{0.1\linewidth}p{0.1\linewidth}}}%
   {\end{longtabu}}%% Set the table’s header: title row
  %\renewcommand*{\glossaryheader}{%
  % \bfseries Term & \bfseries Description & 
  % \bfseries Units & \bfseries Page List
  % \\\endhead}%
  % No table header:
  \renewcommand*{\glossaryheader}{}%
  % No heading between groups:
   % \renewcommand*{\glsgroupheading}[1]{\textbf{\glsgetgrouptitle{##1}}\\}%
  % Main (level 0) entries displayed in a row optionally numbered:
   \renewcommand*{\glossaryentryfield}[5]{%
      \glstarget{##1}{##2}% Name
      &
      & ##3% Description
      % & \glsentryuseri{##1}% Units
      % & ##5% Page list
      \\\vspace{2mm}% end of row
   }%
  % Similarly for sub-entries (no sub-entry numbers):
  \renewcommand*{\glossarysubentryfield}[6]{%
      % ignoring first argument (sub-level)
      &
      \glstarget{##2}{##3}% Name
      & ##4% Description
      % & \glsentryuseri{##2}% Units
      % & ##6% Page list
      \\\vspace{1mm}% end of row
   }%
  % Nothing between groups:
  \renewcommand*{\glsgroupskip}{\indexspace}%
}

% Change the font used by entry names
% \renewcommand{\glsnamefont}[1]{\textbf{#1}}
% \renewcommand{\glsnamefont}[1]{\textsc{\mdseries #1}}
% \renewcommand{\glsnamefont}[1]{\mdseries #1}

% \newglossary[slg]       {symbols}       {not}       {ntn}       {Symbols}
% \setglossarystyle{tree}
\newglossary[modg]      {model}         {modd}      {modn}      {Model}
\newglossary[unitg]     {unit}          {unitd}     {unitn}     {Unit}
% \newglossary[pgmg]      {powergridmodel}{pgmd}      {pgmn}      {Power Grid Model}
% \newglossary[wfmg]      {windfarmmodel} {wfmd}      {wfmn}      {Wind Farm Model}

% Generate the glossary
\makeglossaries

%%%%%%%%%%%%%%%%%%%%%%%%%%%%%%%%%%%%%%%%%%%%%%%%%%%%%%%%%%%%%%%%%%%%%%
%% TikZ stuff

\usetikzlibrary{arrows}
\usetikzlibrary{decorations.text}
\usetikzlibrary{decorations.pathmorphing}
\usetikzlibrary{through}
\usetikzlibrary{intersections}
\tikzset{>=stealth'}

\tikzstyle{figure}=[font=\small]
\tikzstyle{vertex}=[circle, inner sep=0pt, minimum size=6pt, draw=black, fill=primarycolor-vlight]
\tikzstyle{bigvertex}=[vertex, minimum size=11pt]
\tikzstyle{chvertex}=[bigvertex]
\tikzstyle{textvertex}=[bigvertex, font=\footnotesize]
\tikzstyle{highlightedvertex}=[vertex, fill=secondarycolor-light]
\tikzstyle{highlightedchvertex}=[chvertex, fill=secondarycolor-light]
\tikzstyle{chargingstationvertex}=[double distance=1pt, fill=secondarycolor-light]
\tikzstyle{vertexhull}=[line width=14pt, rounded corners=2pt]
\tikzstyle{vertexhullboundary}=[vertexhull, line width=14.8pt]
\tikzstyle{chshortcutedges}=[edges, very thick, fill=secondarycolor, draw=primarycolor]
\tikzstyle{cell1vertex}=[vertex, fill=black30]
\tikzstyle{cell2vertex}=[vertex, fill=black15]
\tikzstyle{cell3vertex}=[vertex, fill=black50]
\tikzstyle{dualvertex}=[vertex, fill=secondarycolor-light]
\tikzstyle{dualtextvertex}=[textvertex, fill=secondarycolor-light]

\tikzstyle{edgelabel}=[inner sep=1pt, fill=white]
\tikzstyle{edges}=[shorten >=1pt, shorten <=1pt]
\tikzstyle{shortestpathedges}=[edges, line width=3pt, fill=secondarycolor-light, draw=secondarycolor-light]
\tikzstyle{internaledges}=[edges]
\tikzstyle{boundaryedges}=[internaledges, thick, densely dotted]
\tikzstyle{edgesbackground}=[line width=2pt, draw=white]
\tikzstyle{shortestpathedgesbackground}=[line width=4.5pt, draw=white]
\tikzstyle{dualedges}=[edges,thick,secondarycolor]

\tikzstyle{polygonalpath}=[thick, draw=primarycolor-dark]
\tikzstyle{polygonwindow}=[very thick,densely dotted]
\tikzstyle{polygonalarea}=[fill=primarycolor-vlight,draw=primarycolor-vlight]
\tikzstyle{polygon}=[polygonalpath, fill=primarycolor-vlight]
\tikzstyle{polygonhole}=[polygonalpath, fill=white]
\tikzstyle{outerpolygonalpath}=[thick]
\tikzstyle{triangulationedge}=[draw=black50]
\tikzstyle{point}=[circle, inner sep=0pt, minimum size=3pt, draw=black, fill=black]
\tikzstyle{specialpoint}=[point,fill=primarycolor]
\tikzstyle{minlinkpath}=[primarycolor,line width=2pt]
\tikzstyle{minlinkshortestpath}=[secondarycolor,very thick]
\tikzstyle{minlinkshortestpathmid}=[thesisyellow,very thick]
\tikzstyle{visibilitycone}=[primarycolor,very thick]
\pgfdeclareverticalshading{myshading}{100bp}{color(0bp)=(primarycolor-light); color(40bp)=(primarycolor-light); color(70bp)=(white); color(100bp)=(white)}
\pgfdeclareverticalshading{myshadingother}{100bp}{color(0bp)=(primarycolor-light); color(10bp)=(primarycolor-light); color(54bp)=(white); color(100bp)=(white)}
\tikzstyle{visibilityconearea}=[bottom color=primarycolor-light, top color=white,draw=white]
\tikzstyle{filledregionprimary}=[primarycolor-vlight]
\tikzstyle{filledregionsecondary}=[secondarycolor-vlight]

\tikzstyle{toplevelpartitionboundary}=[primarycolor-dark,line width=2.25pt]
\tikzstyle{botlevelpartitionboundary}=[secondarycolor-light, line width=1pt]
\tikzstyle{toplevelpartitiondescription}=[text=primarycolor-dark]
\tikzstyle{botlevelpartitiondescription}=[font=\footnotesize,text=secondarycolor-light]

\tikzstyle{function}=[draw=primarycolor,very thick,line cap=rect]
\tikzstyle{functionterm}=[text=primarycolor,edgelabel]
\tikzstyle{discontinuity}=[circle, inner sep=0pt, minimum size=3.5pt,very thick]
\tikzstyle{discontinuityblank}=[discontinuity,draw=primarycolor,fill=white]
\tikzstyle{discontinuityfilled}=[discontinuity,draw=primarycolor,fill=primarycolor]
\tikzstyle{plotgrid}=[color=black30,very thin,dash pattern=on 2pt off 1pt]
\tikzstyle{chargingfunction}=[function,draw=secondarycolor]
\tikzstyle{helperline}=[black]
\tikzstyle{breakpoint}=[fill,primarycolor,circle,inner sep=0pt,minimum size=3.5pt,very thick]

\tikzstyle{shrinkedface}=[thick,double distance=10pt,draw=primarycolor-dark, double=primarycolor-vlight]
\tikzstyle{reachablevertex}=[vertex, fill=white]
\tikzstyle{unreachablevertex}=[vertex, fill=black50]
\tikzstyle{passableedges}=[edges,black]
\tikzstyle{borderedges}=[edges,thick,densely dotted]
\tikzstyle{unreachableedges}=[edges,black]
\tikzstyle{reachableboundary}=[draw=primarycolor,line width=3pt]
\tikzstyle{unreachableboundary}=[draw=secondarycolor,line width=3pt]
\tikzstyle{visitededges}=[edges, line width=3pt, primarycolor-light]

\tikzstyle{plotpoint}=[fill=black,circle,inner sep=0pt,minimum size=3.5pt]
\tikzstyle{tangentline}=[shorten >=-20pt,shorten <= -15pt]

\tikzstyle{markSign} = [mark=*]
\tikzstyle{heuristicMarkSign} = [mark=o]
\tikzstyle{exactMarkSign} = [mark=*]
\tikzstyle{shortenLines} = [shorten <= 3.5pt,shorten >= 3.5pt]

\pgfdeclarelayer{background}
\pgfdeclarelayer{foreground}
\pgfsetlayers{background,main,foreground}

\pgfplotsset{
  /pgfplots/legend line with mark/.style={
    legend image code/.code={
        \draw[##1,no markers,shorten >= 3.5pt,line width=1pt]
         plot coordinates {
         (0cm,0cm)
         (0.3cm,0cm)
        };
        \draw[##1,only marks,markSign,line width=1pt]
         plot coordinates {
         (0.3cm,0cm)
        };
        \draw[##1,no markers,shorten <= 3.5pt,line width=1pt]
         plot coordinates {
         (0.3cm,0cm)
         (0.6cm,0cm)
        };
    }
  }
}

\pgfplotsset{
  /pgfplots/legend line with heuristic mark/.style={
    legend image code/.code={
        \draw[##1,no markers,shorten >= 3.5pt,line width=1pt]
         plot coordinates {
         (0cm,0cm)
         (0.3cm,0cm)
        };
        \draw[##1,only marks,heuristicMarkSign,line width=1pt]
         plot coordinates {
         (0.3cm,0cm)
        };
        \draw[##1,no markers,shorten <= 3.5pt,line width=1pt]
         plot coordinates {
         (0.3cm,0cm)
         (0.6cm,0cm)
        };
    }
  }
}

\pgfplotsset{
  /pgfplots/legend line with exact mark/.style={
    legend image code/.code={
        \draw[##1,no markers,shorten >= 3.5pt,line width=1pt]
         plot coordinates {
         (0cm,0cm)
         (0.3cm,0cm)
        };
        \draw[##1,only marks,exactMarkSign,line width=1pt]
         plot coordinates {
         (0.3cm,0cm)
        };
        \draw[##1,no markers,shorten <= 3.5pt,line width=1pt]
         plot coordinates {
         (0.3cm,0cm)
         (0.6cm,0cm)
        };
    }
  }
}

\newcommand{\substationsymbol}{
\protect\tikz \protect
\begin{tikzpicture}
\filldraw[fill=orange,draw=orange] rectangle (1pt,1pt);
\end{tikzpicture}
}

%%%%%%%%%%%%%%%%%%%%%%%%%%%%%%%%%%%%%%%%%%%%%%%%%%%%%%%%%%%%%%%%%%%%%%
%% Misc settings

\widowpenalties 2 10000 0
\clubpenalties 2 10000 0

\hyphenation{Karls-ruhe}
\hyphenation{Lem-ma}
\hyphenation{Open-Access}
\hyphenation{Trans-for-ma-tion}
\hyphenation{ca-pa-ci-ty}
\hyphenation{mul-ti-ple}
\hyphenation{trap-ped}
\hyphenation{para-dox}
\hyphenation{ap-prox-i-ma-tion}
\hyphenation{switch-es}
\hyphenation{poly-no-mi-al}
\hyphenation{re-pre-sen-ta-ti-on}
\hyphenation{so-lu-ti-on}
\hyphenation{tri-sub-graph}

\setlength{\tabcolsep}{1.0ex}

\providecommand\phantomsection{}