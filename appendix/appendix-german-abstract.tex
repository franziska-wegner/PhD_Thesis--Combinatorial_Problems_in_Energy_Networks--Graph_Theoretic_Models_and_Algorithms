\selectlanguage{ngerman}
\chapter{Deutsche Zusammenfassung}
% 
Energienetze bilden das R{\"u}ckgrat unserer Gesellschaft, die unter anderem
unsere Nahrungskette und andere wichtige Infrastrukturen, wie die Wasser- und
W{\"a}rmeversorgung, bestimmen. Um die grundlegenden menschlichen
Bed{\"u}rfnisse zu befriedigen, m{\"u}ssen wir ein nachhaltigeres und
umweltfreundlicheres Verhalten im Allgemeinen und in Energienetzen im Speziellen
an den Tag legen. In dieser Arbeit geht es um Energienetze, wobei wir uns auf
Stromnetze spezialisieren und uns darauf fokussieren, wie wir die vorhandene
Infrastruktur besser ausnutzen k{\"o}nnen. Wir merken an, dass die Ergebnisse
aus dieser Arbeit auch auf andere Energienetze {\"u}bertragen werden
k{\"o}nnen~\parencite{Gro19} und bestimmte auftretende Ph{\"a}nomene legen es
nahe, dass sich einige Ergebnisse eventuell auch auf Verkehrsnetze
{\"u}bertragen lassen. Diese Arbeit besteht aus vier inhaltlichen Teilen. Der
erste Teil besch{\"a}ftigt sich mit der Funktionsweise und Struktur von
elektrischen Fl{\"u}{ss}en. Der zweite und dritte inhaltliche Teil der Arbeit
besch{\"a}ftigt sich jeweils mit der effizienten Ausnutzung der vorhandenen
Energienetzinfrastruktur. Dabei verstehen wir hier unter effizienter Ausnutzung
entweder die Maximierung der Gesamterzeugung und die damit verbundene
Erweiterung des Betriebspunktes oder die Minimierung der Erzeugungskosten
verstehen.

Das elektrische Netz besteht aus drei Spannungsebenen, die wir als Hoch-,
Mittel-, und Niederspannungsebene bezeichnen. Das traditionelle elektrische Netz
ist auf eine zentrale Energieversorgung ausgelegt, bei der die Erzeuger sich in
der Hochspannungsebene befinden. Der elektrische Fluss im klassischen Sinne
flie{\ss}t von der Hoch- in die Mittel- und Niederspannungsebene. Die
industriellen Verbraucher befinden sich zumeist auf der Mittelspannungsebene,
w{\"a}hrend sich die Haushalte und kleineren Industrien in der
Niederspannungsebene befinden. Durch nachhaltige Erzeuger, die ihre Energie aus
erneuerbaren Energien wie beispielsweise Wind gewinnen, findet nun ein
Paradigmenwechsel im elektrischen Netz statt. Diese nachhaltigen Erzeuger
befinden sich zumeist im Nieder- und Mittelspannungsnetz und der elektrische
Fluss k\"onnte nun bidirektional flie{\ss}en. Dieser Paradigmenwechsel kann zu
Engp\"a{\ss}en und anderen Problemen f{\"u}hren, da das elektrische Netz f\"ur
ein solches Szenario nicht konzipiert ist.

Eine Hauptaufgabe dieser Arbeit war die Identifizierung von Problemstellungen in
elektrischen Netzen. Die extrahierten Problemstellungen haben wir dann in
graphen-theoretische Modelle {\"u}bersetzt und Algorithmen entwickelt, die
oftmals G{\"u}tegarantien besitzen. Wir haben uns dabei zun{\"a}chst auf die
Modellierung von elektrischen Netzen und das Verhalten von Fl{\"u}ssen in diesen
Netzen mit Hilfe von Graphentheorie konzentriert. Zur Modellierung des
elektrischen Flusses nutzen wir eine linearisierte Modellierung, die mehrere
vereinfachende Annahmen trifft. Diese linearisierte Modellierung ist f{\"u}r
Hochspannungsnetze im Allgemeinen eine gute Ann{\"a}herung und macht das
Entscheidungsproblem f{\"u}r elektrische Fl{\"u}sse, das hei{\ss}t, ob ein
g{\"u}ltiger elektrischer Fluss f{\"u}r eine bestimmte Konfiguration des Netzes
und f{\"u}r einen bestimmten Verbrauch und eine bestimmte Erzeugung existiert,
in Polynomialzeit l{\"o}sbar.

\textbf{Leistungsfluss.} Fokusiert man sich auf das vereinfachte
Zul{\"a}ssigkeitsproblem von elektrischen Fl\"ussen und den Ma\-xi\-ma\-len
Leistungsfl{\"u}ssen, so existieren verschiedene mathematische Formulierungen,
die den Leistungsfluss beschreiben. Auf allgemeinen Graphen ist es oftmals der
Fall, dass graphentheoretischen Fl{\"u}sse keine zul{\"a}ssigen
Leistungsfl{\"u}sse darstellen. Im Gegensatz zu graphentheoretischen Fl{\"u}ssen
balancieren sich Leistungsfl{\"u}sse. Wir diskutieren diese Eigenschaft aus
% 
graphentheoretischer Sicht. Die verschiedenen mathematischen Formulierungen
geben uns strukturelle Einblicke in das Lei\-stungs\-fluss\-problem. Sie zeigen
uns die Dualit{\"a}t der zwei Kirchhoffschen Regeln. Diese nutzen wir um einen
algorithmischen Ansatz zur Berechnung von Leistungsflüssen zu formulieren, der
zu einem Algorithmus für Leistungsflüsse auf planaren Graphen führen k{\"o}nnte.
Die Einschr{\"a}nkung auf planare zweifachzusammenh{\"a}ngende Graphen ist
vertretbar, da elektrische Netze im Allgemeinen planar sind~\parencite[S.\
13]{Cai12}. Zudem hilft uns diese Sichtweise, um Analogien zu anderen
geometrischen Problemen herzustellen.

\textbf{Kontinuierliche {\"A}nderungen.} Da graphentheoretische Fl{\"u}sse sich
in vielen F{\"a}llen anders als elektrische Fl{\"u}sse verhalten, haben wir
versucht, das Stromnetz mittels Kontrolleinheiten so auszustatten, dass der
elektrische Fluss den gleichen Wert hat wie der graphentheoretische Fluss. Um
dieses Ziel zu erreichen, platzieren wir die Kontrolleinheiten entweder an den
Knoten oder an den Kanten. Durch eine Suszeptanz-Skalierung, die durch die
Kontrolleinheiten erm{\"o}glicht wird, ist es nun prinzipiell m{\"o}glich jeden
graphentheoretischen Fluss elektrisch zul{\"a}ssig zu machen. Dabei konnten wir
zeigen, dass das gezielte Platzieren von Kontrolleinheiten die Kosten der
Erzeugung von elektrischer Leistung durch Generatoren im elektrischen Netz
senken kann und den Betriebspunkt des Netzes in vielen F{\"a}llen auch
erweitert. Platziert man Kontrolleinheiten so, dass der verbleibende Teil (d.\
h.\ das Netz ohne die Kontrolleinheiten) ein Baum oder Kaktus unter geeigneter
Begrenzung der Kapazit{\"a}ten ist, so ist es m{\"o}glich, jeden
graphentheoretischen Fluss als elektrisch zul{\"a}ssigen Fluss mit
gleichwertigen Kosten zu realisieren. Die Kostensenkung und die Erweiterung des
Betriebspunktes konnten wir experimentell auf~\gls{ieee}-Benchmark-Daten
best{\"a}tigen.

\textbf{Diskrete {\"A}nderungen.} Die oben beschriebenen Kontrolleinheiten sind
eine idealisierte, aktuell nicht realisierbare Steuereinheit, da sie den
elektrischen Fluss im gesamten Leistungsspektrum einstellen können. Damit ist
vor allem gemeint, dass sie den elektrischen Fluss auf einer Leitung von \glqq
Die Leitung ist abgeschaltet.\grqq\ bis zur maximalen Kapazit{\"a}t stufenlos
einstellen k{\"o}nnen. Diese Idealisierung ist auch ein
% 
gro{\ss}er Kritikpunkt an der Modellierung. Aus diesem Grund haben wir versucht,
unser Modell realistischer zu gestalten. Wir haben zwei m{\"o}gliche
Modellierungen identifiziert. In der ersten Modellierung können Leitungen ein-
und ausgeschaltet werden. Dieser Prozess wird als Switching bezeichnet und kann
in realen Netzen mittels Circuit Breakers (dt.\ Leistungsschaltern) realisiert
werden. Die zweite Modellierung kommt der Kontrolleinheiten-Modellierung sehr
nahe und beschäftigt sich mit der Platzierung von~Kontrolleinheiten, die die
Suszeptanz innerhalb eines gewissen Intervalls einstellen k{\"o}nnen. Diese
wirkt im ersten Moment wie eine Verallgemeinerung der
Schaltungsflussmodellierung. Nutzt man jedoch eine realistischere Modellierung
der Kontrolleinheiten, so ist das Einstellen der Suszeptanz durch ein Intervall
begrenzt, das das Ausschalten einer Leitung nicht mit beinhaltet. Sowohl ein
optimales (im Sinne der Minimierung der Gesamterzeugungskosten oder der
Maximierung des Durchsatzes) Platzieren von Switches als auch ein optimales
Platzieren von~Kontrolleinheiten ist im Allgemeinen
\NP-schwer~\parencite{Leh14}. Diese beiden Probleme erg{\"a}nzen sich
dahingehend, dass man den maximalen graphentheoretischen Fluss, mit den zuvor
genannten Platzierungen ann{\"a}hern kann.

F{\"u}r Switching konnten wir zeigen, dass das Problem bereits schwer ist, wenn
der Graph serien-parallel ist und das Netzwerk nur einen Erzeuger und einen
Verbraucher besitzt~\parencite{Gra18}. Wir haben sowohl für den Maximalen
{\"U}bertragungsschaltungsfluss (engl.\ Maximum Transmission Switching Flow;
kurz~\gls{mtsf}) als auch für den optimalen {\"U}bertragungsschaltungsfluss
(engl.\ Optimal Switching Flow; kurz~OSF) erste algorithmische Ans{\"a}tze
vorgeschlagen und gezeigt, dass sie auf bestimmten graphentheoretischen
Strukturen exakt sind, und dass auf anderen graphentheoretischen Strukturen
G{\"u}tegarantien m{\"o}glich sind~\parencite{Gra18}. Die Algorithmen haben wir
dann auf allgemeinen Netzen evaluiert. Simulationen führen zu guten Ergebnissen
auf den~\gls{nesta}-Benchmark-Daten.

\textbf{Erweiterungsplanung auf der Gr{\"u}nen Wiese.} Eine vom Rest der 
Arbeit eher losgel{\"o}ste Fragestellung war die Verkabelung von Windturbinen.
Unter Verwendung einer Metaheuristik haben wir gute Ergebnisse im Vergleich zu
einem \glqq Mixed Integer Linear Program\grqq\ (\gls{milp}; dt.\
gemischt-ganzzahliges lineares Programm) erzielt, das wir nach einer Stunde
abgebrochen haben. Die Modellierung der Problemstellung und die Evaluation des
Algorithmus haben wir auf der ACM e-Energy 2017
ver{\"o}ffentlicht~\parencite{Leh17}.

\textbf{Schlusswort.} Abschlie{\ss}end kann man sagen, dass mit dieser Arbeit
allgemeine, tiefliegende Aussagen {\"u}ber elektrische Netze getroffen wurden,
unter der Ber{\"u}cksichtigung struktureller Eigenschaften unterschiedlicher
Netzklassen. Diese Arbeit zeigt wie das Netz ausgestaltet sein muss, um
bestimmte Eigenschaften garantieren zu k{\"o}nnen und zeigt verschiedene
L{\"o}sungsans{\"a}tze mit oft beweisbaren G{\"u}tegarantien auf. 