\oldchapter*{Abstract}
% 
In this thesis, we study combinatorial problems in energy networks with the
focus on power grids. At present we see a paradigm shift in power grids towards
renewable energy, while making use of the traditional power grid. This shift
changes the production pattern from a centralized way towards a distributed
production, leading to bottlenecks and other problems. We try to efficiently
exploit the existing infrastructure by analyzing the structure of and developing
algorithms for electrical flows, placement problems, and layout problems to
improve the existing power grid. We remark that the results of this work might
be applicable to other energy networks as well~\parencite{Gro19} and certain
phenomena such as the Braess's Paradox (\ie, for road network it means that
adding a road to the traffic network can cause longer travel times) indicate
that the provided techniques in this thesis could be used for traffic networks,
too.

One main task of this work was the identification of problem statements in
energy networks. We first translate the problems to graph-theoretical models
such that we are able to analyze the problems, study their complexity, develop
algorithms, and evaluate them using either existing data sets or generated data
if there are no publicly available suitable data sets. We develop algorithms
that provide in most cases quality guarantees on certain graph classes that can
be then used as good heuristics on general graphs. At first we focus on the
modeling of power grids and the behavior of electrical flows in power grids
using a linearized model that makes use of some simplifications. These
simplifications are based on realistic assumptions for high-voltage power grids
on which we lay our focus.

This thesis has four main content chapters. The first part focuses on algorithms
to compute electrical flows. We describe the mathematical structure and focus on
some major properties of electrical flows. Note that apart from solving a system
of linear equations or an exponential time algorithm there are no known
algorithms to compute electrical flows. One way to tackle this problem are
electrical preserving transformations. Electrical preserving transformations are
common techniques in the electrical flow analysis. Based on these
transformations, we will present a first algorithm for electrical flows
on~\source-\sink-planar biconnected graphs. In addition to that, we discuss
different representations and formulations of electrical flows that increase the
understanding of the electrical flow's behavior. We make use of these
representations to describe the balancing property by separating the quadratic
relationship of voltage and current. This leads us to the duality of the two
Kirchhoff laws and another algorithmic approach.

The second and third part of this thesis focus on the increasing of the
efficiency of the electrical network. We exploit the Braess' Paradox by
switching lines (\ie, temporarily removal of a line or cable) or by using an
edge weight scaling (\ie, susceptance scaling). We design novel algorithms that
improve the throughput of the power grid or decrease the overall operating
costs. These algorithms are the first that provide some quality guarantees or
bounds. Each of these parts includes simulations to evaluate the algorithms on a
realistic data set.

The last part of this thesis is about transmission network expansion planning on
a greenfield motivated by the wind farm cabling problem. Algorithmically, it
represents a layout problem. Within this part, we present a first proper model
formulation for this particular problem, give a benchmark generator, and design
a meta-heuristic approach to tackle the wind farm cabling problem that is then
evaluated on a generated data set.
% 