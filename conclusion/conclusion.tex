%%%%%%%%%%%%%%%%%%%%%%%%%%%%%%%%%%%%%%%%%%%%%%%%%%%%%%%%%%%%%%%%%%%%%%%%%%%%%%%%
\chapter[Conclusion]{Conclusion}\label{ch:conclusion}
%%%%%%%%%%%%%%%%%%%%%%%%%%%%%%%%%%%%%%%%%%%%%%%%%%%%%%%%%%%%%%%%%%%%%%%%%%%%%%%%
% 
In this thesis, we identified a bunch of problems in energy networks. We
analyzed the problems and networks such that we were able to design algorithms
that give certain guarantees or bounds. In addition, we evaluated the algorithms
on the~\gls{ieee} benchmark data set or on self-generated instances. In this
chapter, we briefly recapitulate the main results of this thesis
in~\cref{ch:conclusion:sec:summary}. There are still many ideas and open
questions, which we were not able to tackle during that time. Thus, we give an
outlook of what can be done in the future by outlining a few of the remaining
ideas and open problems in~\cref{ch:conclusion:sec:outlook}.
% 
%%%%%%%%%%%%%%%%%%%%%%%%%%%%%%%%%%%%%%%%%%%%%%%%%%%%%%%%%%%%%%%%%%%%%%%%%%%%%%%%
\section{Summary}\label{ch:conclusion:sec:summary}
%%%%%%%%%%%%%%%%%%%%%%%%%%%%%%%%%%%%%%%%%%%%%%%%%%%%%%%%%%%%%%%%%%%%%%%%%%%%%%%%
%
One of the most fundamental problems that is part of nearly all problems that
incorporate the power grid is the feasibility problem for electrical flows. We
gave a first comprehensive analysis of electrical flows. We showed the duality
of the two Kirchhoff's laws known as~\acrlong{kcl} and~\acrlong{kvl} that
separates the relationship between current and voltage by using two graphs. We
also show different possible representations that increase the understanding of
electrical flows leading to different properties in electrical networks. We
think that one of the most interesting properties is the balancing property that
basically shows that all paths from one vertex to another vertex have the same
length.
% 
We developed first algorithms for the feasibility problem on
\source-\sink-planar biconnected graphs that can be seen as a \source-\sink
electrical flow decomposition. Using the superposition principle, we can compute
all \source-\sink electrical flows and combine them into one electrical flow.
The algorithmic idea that uses the duality has the potential to be used in
dynamic scenarios, since a topology change does not need a full new
recomputation of the flow, but can start from an already existing solution.

The second content chapter is about~\acrlong{mtsf} (\gls{mtsf}) that can be seen
as a discrete manipulation of the electrical network topology. We are the first
that develop algorithms with provable guarantees on certain graph structures and
shrink the gap between theory and practice. With the theoretical analysis of
this problem, we are able to build connections to related problems. We showed
network simplifications including transformations from the bounded to the
unbounded~\gls{mtsf} and the equivalence between~\acrlong{ots} (\gls{ots})
and~\gls{mtsf}. We introduced exact algorithms for networks with one generator
and one demand for certain graph structures. We also show when the problem
becomes already \NP-hard on~\source-\sink networks. Though the algorithms are
only designed for special graph structures, we evaluate them on general power
grids. For the \source-\sink algorithm, which we called~\acrlong{dtp}
(\gls{dtp}), we defined a new centrality. The results of the centrality seem to
give a hint on which edges are critical, since there are edges where the
electrical network degenerates and we realized that these edges are the ones
with a low centrality.

The third chapter is about the continuous manipulation of the electrical network
topology. We motivate this by placing~\gls{facts} either at vertices or at
edges. For that problem, we present a hybrid model for including flow control
vertices or edges. We were able to show that it suffices to place a relatively
small number of flow control units to reach the same solution as the
graph-theoretical flow solution that is equivalent to placing these units
everywhere. In addition to that, we even saw that fewer flow control buses
improve the loadability and even have lower cost increase compared
to~\acrlong{opf} (\gls{opf}). We were able to explain our empirical observations
on controller placement with graph-theoretical means.

We focused in the last part of this work on transmission network expansion
planning on the green field. This particular problem represents a layout problem
and was motivated by the wind farm cabling problem. We assumed that the turbines
have fixed positions and that there are multiple cables types. We want to find a
cabling of the wind farm such that the overall cabling costs are minimized. This
is in general an~\NP-hard problem. We give a first proper model and decompose the
wind farm cabling problem into multiple subproblems, each remaining~\NP-hard for
multiple cable types. We developed a meta-heuristic known as~\acrlong{sa}
algorithm, which we compare to the~\gls{milp} by using various benchmark sets
that we generated to enable comparability and to overcome the shortcoming of the
current literature. We could see good results in the simulations for medium to
large wind farms.


%%%%%%%%%%%%%%%%%%%%%%%%%%%%%%%%%%%%%%%%%%%%%%%%%%%%%%%%%%%%%%%%%%%%%%%%%%%%%%%%
\section{Outlook}\label{ch:conclusion:sec:outlook}
%%%%%%%%%%%%%%%%%%%%%%%%%%%%%%%%%%%%%%%%%%%%%%%%%%%%%%%%%%%%%%%%%%%%%%%%%%%%%%%%

For the electrical flow feasibility problem there remain several further
investigations and proofs to explain the properties of such an electrical flow
in depth. Furthermore, it would be interesting to evaluate whether the
assumptions we make---meaning biconnectivity and planarity---are reasonable
assumptions for the problem. We think that planarity should be a reasonable
assumption following the statement of~\textcite[p.13]{Cai12} and even the
biconnectivity assumption for an~\source-\sink subgraph should be reasonable,
since the electrical flow takes a path with the least resistance. However, it
might lead to some error, which will be interesting to evaluate in simulations.

For~\acrlong{mtsf} (\gls{mtsf}), which is a discrete manipulation of the power
grid topology, it is unknown to us whether the reachability test can be done in
polynomial time and if not, whether there is a polynomial time algorithm that
finds all~\acrlong{dtp}s (\gls{dtp}s) from one source~\source to one sink~\sink.
Another open question to us is whether there is a~\acrlong{ptas} (\gls{ptas}) on
cacti. It would be also interesting to see whether the complexity changes and
whether there are algorithms when we define a set of edges as non-switchable
(motivated by~\gls{tnep}). Other interesting problems are the minimization and
the constraining of the number of switches.

For the manipulation of the power grid topology using control units a complexity
analysis would give the problem more structure and would increase the
understanding of the problem. It would be even interesting if we can adopt the
methods and algorithm from~\gls{mtsf} such that they work for the susceptance
scaling, too. It is also unknown to us if the bilinearity will help us to some
extent.

For both problems it would be interesting if the recent findings in the
electrical flow feasibility also help in developing better algorithms for the
discrete and continuous manipulation in power grids. In addition, a comparison
to more realistic network models such as~\gls{ac} model is another main
evaluation.

For the wind farm cabling there are many open questions concerning complexity
and algorithms that give certain guarantees. In addition, so far we completely
omit electrical flows in the wind farm cabling, since we more or less assume
that the resulting networks are tree-like and thus, a graph-theoretical flow is
a reasonable assumption. It would be worth investigating whether the results are
electrically feasible and to incorporate electrical flows in general.
% 