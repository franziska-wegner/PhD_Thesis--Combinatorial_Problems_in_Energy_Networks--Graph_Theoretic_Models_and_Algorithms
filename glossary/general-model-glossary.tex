\newglossaryentry{gurobi}
{
   name={Gurobi Optimizer},
   longform={Gurobi},
   capitalform={Gurobi Optimizer},
   sort={gurobi},
   description={Gurobi is a solver that can solve optimization problems
   formulated as an~\acrlong{lp} (\gls{lp}) or an~\gls{milp} (\gls{milp}).}
}



\newglossaryentry{qt}
{
    name={QT},
    longform={Quality Threshold},
    capitalform={Quality Threshold},
    sort={qt},
    description={\glslongform{qt} is a clustering algorithm.},
    symbol={QT}
}


\newglossaryentry{problems}
{
    name={\ensuremath{\problems}},
    longform={problems},
    capitalform={Problems},
    sort={instances},
    % 
    description={ With the problem~\glssymbol{problems}, we abstract from
        specific problems such as~\acrlong{sppp} (\gls{sppp}). },
    % 
    symbol={\ensuremath{\problems}}
}



\newglossaryentry{pu}
{
    name={p.u.},
    longform={per-unit-system},
    capitalform={Per-unit-system},
    sort={pu},
    description={The per-unit-system (p.u.) is a normalization system used in
    the power system analyzes to simplify the calculation of the power flow
    while having heterogeneous networks.},
    symbol={p.u.} 
}

% activity threshold
% substation capacity tightness - ratio of maximum supply and demand

% Mathematical Model
\newglossaryentry{graph}
{
    name={\graph},
    sort={graph},
    type={model},
    % 
    description={ The topological structure of a power grid is a
    graph~$\glssymbol{graph}
    =
    (\glssymbol{vertices},\glssymbol{edges})$, where the
    set~\glssymbol{vertices} is the set of vertices and the
    set~\glssymbol{edges} is the set of edges. },
    %
    % Bidirected graph.
    symbol={\graph}
}

\newglossaryentry{embedding}
{
    name={\embedding},
    sort={embedding},
    parent={graph},
    type={model},
    % 
    description={A fixed planar embedding~\glssymbol{embedding} of a
    graph~\glssymbol{graph} into the plane
    with~$\glssymbol{graph}(\embedding)\cong\glssymbol{graph}$ (\ie,
    $\glssymbol{graph}(\embedding)$ is isomorphic to~\glssymbol{graph}) and a
    injective function~$\embeddingMap\colon
    \glssymbol{vertices}\to\reals\times\reals$ meaning there is a one-to-one
    correspondence between the vertices~\glssymbol{vertices} of the graph and
    the geometrical points~$\points$ of the plane embedding. },
    % 
    symbol={\embedding}
}

\newglossaryentry{dualgraph}
{
    name={\dualgraph},
    sort={dualgraph},
    parent={graph},
    type={model},
    % 
    description={ A dual graph~\glssymbol{dualgraph} of a planar
    graph~\glssymbol{graph} with a fixed planar embedding~\glssymbol{embedding}
    is a graph that has for each face of~\glssymbol{graph} a vertex and whenever
    two faces are incident to each other in graph~\glssymbol{graph} these two
    vertices representing the faces are connected by an edge in the dual
    graph~\glssymbol{dualgraph}. There is a one-to-one correspondence between
    the edges of the primal graph~\glssymbol{graph} and the edges of the dual
    graph~\glssymbol{dualgraph}. },
    % 
    symbol={\dualgraph}
}

\newglossaryentry{wyedeltagraph}
{
    name={\wyedelta graph},
    sort={graph-wye-delta},
    parent={graph},
    type={model},
    % 
    description={The \glssymbol{wyedelta} graph is a topological structure of
    the power grid that can be reduced to a vertex by the following reduction
    rules: degree 1 and self-loop deletion, series and parallel contraction, and
    \glssymbol{wyedelta} transformations.
    % 
    },
    % 
    symbol={\ensuremath{\graph_{\wyedelta}}}
}

\newglossaryentry{deltawye}
{
    name={\deltawye},
    sort={deltawye},
    longform={delta-wye},
    capitalform={Delta-wye},
    type={model},
    % 
    description={
    % 
    The delta-wye transformation transforms a triangle to a star by adding one
    vertex into the center and adding edges from the center to the already
    existing vertices.
    % 
    },
    % 
    symbol={\ensuremath{\deltawye}}
}

\newglossaryentry{wyedelta}
{
    name={\wyedelta},
    sort={wyedelta},
    longform={wye-delta},
    capitalform={Wye-delta},
    type={model},
    % 
    description={
    % 
    The wye-delta transformation transforms a star with three edges to a
    triangle by removing the star's center and its incident edges, and adding
    edges such that the remaining vertices build a complete graph that is a
    triangle.
    % 
    },
    % 
    symbol={\ensuremath{\wyedelta}}
}

\newglossaryentry{cubicgraph}
{
    name={cubic graph},
    sort={cubic graph},
    parent={graph},
    type={model},
    % 
    description={A cubic graph (also known as trivalent graph) is a 3-regular
    graph, where all vertices have a degree of three.
    % 
    },
    symbol={}
}

\newglossaryentry{3attachedsubgraph}
{
    name={3-attached subgraph},
    sort={graph-3-attached},
    parent={graph},
    type={model},
    % 
    description={A $3$-attached subgraph is a subgraph of~\glssymbol{graph} with
    three designated vertices~$\glssymbol{vertices}_3$ that are not directly
    connected to each other. In addition, removing~$\glssymbol{vertices}_3$
    from~\glssymbol{graph}---meaning~$\glssymbol{graph}-
    \glssymbol{vertices}_3$---results in at least two connected components one
    of which is called~$H'$. The induced subgraph of~$H'$ and~$\vertices_3$ is
    denoted by~$H$ that consists of an inner part~$H'$ and an outer
    part~$\glssymbol{graph}-H$ that both have at least a size
    of~$\fmagnitude{\glssymbol{vertices}} = 1$. If the size
    of~$\fmagnitude{\glssymbol{vertices}(H')}\geq 2$ it is called
    \emph{non-trivial}.
    % 
    },
    symbol={}
}

\newglossaryentry{trisubgraph}
{
    name={tri\-sub\-graph},
    sort={trisubgraph},
    parent={graph},
    type={model},
    % 
    description={Is a minimal non-trivial~\gls{3attachedsubgraph}.
    % 
    },
    symbol={}
}

\newglossaryentry{network}
{
    name={\network},
    sort={network},
    type={model},
    % 
    description={A network~\glssymbol{network} represents a power grid with 
    electrical parameters such as~$\dcnetworktuple$ for the~\gls{dc}-network.
    See~\cref{ch:foundations} for other example networks.},
    % 
    symbol={\network}
}

\newglossaryentry{vertices}
{
    name={\vertices},
    sort={vertices},
    parent={graph},
    type={model},
    % 
    description={The set~\glssymbol{vertices} of vertices (in the power grid
    denoted as buses) represent points such as transmission line junctions.
    % 
    },
    % 
    symbol={\vertices}
}

\newglossaryentry{generators}
{
    name={\generators},
    sort={generators},
    type={model},
    parent={network},
    % 
    description={The set~\glssymbol{generators} of generator vertices is a
    subset of the set~\vertices of vertices. It represents the set of sources.
    % 
    },
    % 
    symbol={\generators}
}

\newglossaryentry{consumers}
{
    name={\consumers},
    sort={consumers},
    type={model},
    parent={network},
    % 
    description={The set~\glssymbol{consumers} of consumer vertices is a subset
    of the set~\glssymbol{vertices} of vertices. It represents the set of
    sinks.
    % 
    },
    % 
    symbol={\consumers}
}

\newglossaryentry{edges}
{
    name={\edges},
    sort={edges},
    parent={graph},
    type={model},
    % 
    description={The set~\glssymbol{edges} of edges (in the power grid denoted
    as branches) represents curves that interconnect to points such as
    transmission lines or cables.},
    % 
    symbol={\edges}
}

\newglossaryentry{undirectededges}
{
    name={\ensuremath{\protect\undirectededges}},%
    sort={edgesUndirected},
    type={model},
    parent={graph},
    % 
    description={The set~\glssymbol{undirectededges} of undirected edges that is
    represented by unordered pairs of vertices~$
    \{\vertexa,\vertexc\}\in\glssymbol{undirectededges}(\glssymbol{graph})$.},
    % 
    symbol={\ensuremath{\protect\undirectededges}}%
}

\newglossaryentry{capacity}
{
    name={\ensuremath{\capacity}},
    sort={capacity},
    type={model},
    parent={network},
    % 
    description={The capacity is in general a function~$\glssymbol{capacity}
    \colon \glssymbol{undirectededges}\to\posreals$ representing the (thermal
    line) limits of an edge. For the wind farm cabling problem the capacity is a
    function~$
    \glssymbol{capacity} 
    \colon
    \glssymbol{cabletypes} 
    \to 
    \posreals 
    \cup 
    \{\infty\}$.
    % 
    },
    % 
    symbol={\ensuremath{\capacity}}
}

\newglossaryentry{mincapacity}
{
    name={\ensuremath{ \mincapacity }},
    sort={capacityMinimum},
    type={model},
    parent={network},
    description={The minimum capacity on a path~$\glssymbol{path}$.},
    symbol={\ensuremath{ \mincapacity }}
}

\newglossaryentry{gencostfct}
{
    name={\gamma},
    sort={generatorCostFunction},
    type={model},
    parent={network},
    % 
    description={The generator cost function~$\gamma(\vertexa)$ of a generator
    at a vertex~$\vertexa\in\glssymbol{generators}$.
    % 
    },
    % 
    symbol={\gamma}
}

\newglossaryentry{flow}
{
    name={\flow},
    sort={flow},
    type={model},
    % 
    description={The flow~\glssymbol{flow} is a function~$\glssymbol{flow}
    \colon \glssymbol{edges} \to \reals$ satisfying the skew-symmetry property.
    % 
    },
    % 
    symbol={\flow}
}

\newglossaryentry{netflow}
{
    name={\netflow},
    sort={netflow},
    type={model},
    parent={flow},
    % 
    description={The net flow~$\glssymbol{netflow}(\vertexa)$ at a
    vertex~$\vertexa\in\glssymbol{vertices}$ is defined by the sum of all incident
    edges~$\glssymbol{netflow}(\vertexa)\coloneqq\sum_{
    \{\vertexa,\vertexb\}\in\glssymbol{undirectededges}}\glssymbol{flow}
    (\vertexa,\vertexb)$.
    % 
    },
    % 
    symbol={\netflow}
}

\newglossaryentry{flowvalue}
{
    name={\flowvalue},
    sort={flowValue},
    type={model},
    parent={flow},
    % 
    description={The flow
    value~$\glssymbol{flowvalue}(\glssymbol{network},\glssymbol{flow})$ of a
    network~\glssymbol{network} and some flow~\glssymbol{flow} is defined
    by~$\glssymbol{flowvalue}(\glssymbol{network},\glssymbol{flow})\coloneqq\sum_{\vertexa\in\glssymbol{generators}}\glssymbol{netflow}(\vertexa)$.
    % 
    },
    % 
    symbol={\flowvalue}
}

\newglossaryentry{timestamp}
{
    name={\ensuremath{\timestamp}},
    sort={timestamp},
    type={model},
    % 
    description={A timestamp represents a point in time.
    % 
    },
    % 
    symbol={\ensuremath{\timestamp}}
}

\newglossaryentry{switched}
{
    name={\switched},
    sort={switched},
    type={model},
    % 
    description={Set of switched edges~\glssymbol{switched}
    with~$\glssymbol{switched}\subseteq\glssymbol{undirectededges}$. Note that
    for a static analysis with one timestamp the switch is in OFF-state for
    these edges. Roughly speaking the transmission line is temporary removed
    from the topology.
    % 
    },
    % 
    symbol={\switched}
}

\newglossaryentry{switch}
{
    name={\ensuremath{\switch(\vertexa,\vertexb)}},
    sort={switch},
    type={model},
    parent={switched},
    % 
    description={The function~$\switch\colon\undirectededges\to\{0,1\}$ 
    is~$\switch(\vertexa,\vertexb) = 0$ if an edge is switched and~$\switch
    (\vertexa,\vertexb) = 1$ otherwise.
    % 
    },
    % 
    symbol={\ensuremath{\switch}}
}

\newglossaryentry{numberdtps}
{
    name={\ensuremath{\sigma_{\dtp}(\source, \sink, \edge)}},
    sort={dtpnumber},
    type={model},
    % 
    description={The number of~\dtp{s} between~\source and~\sink.},
    % 
    symbol={\ensuremath{\sigma_{\dtp}(\source, \sink, \edge)}}
}

\newglossaryentry{optimalvalue}
{
    name={\ensuremath{\opt_\Pi}},
    sort={optimumValue},
    type={model},
    % 
    description={The optimum value of a problem~$\Pi$~(\eg, \gls{mtsf}) is
    denoted by~\glssymbol{optimalvalue}~(\eg, $\opt_{\gls{mtsf}}$).
    % 
    },
    % 
    symbol={\ensuremath{\opt_\Pi}}
}

\newglossaryentry{switchingcentrality}
{
    name={\ensuremath{ c_{\scu} }},
    sort={centralitySwitching},
    type={model},
    % 
    description={The switching centrality is denoted
    by~$\glssymbol{switchingcentrality}$.
    % 
    },
    % 
    symbol={\ensuremath{ c_{\scu} }}
}

\newglossaryentry{neighbor}
{
    name={\ensuremath{ \neighbor }},
    sort={neighbor},
    type={model},
    % 
    description={The set of neighbors---\ie, the neighborhood---of a
vertex~\vertexb is denoted by~$\glssymbol{neighbor}(\vertexb)=
\{\vertexa\in\glssymbol{vertices}\mid\{\vertexa,\vertexb\}\in
\glssymbol{undirectededges}\}$.
    % 
    },
    % 
    symbol={\ensuremath{ \neighbor }}
}

\newglossaryentry{inneighbor}
{
    name={\ensuremath{ \inneighbor }},
    sort={inneighbor},
    parent={neighbor},
    type={model},
    % 
    description={The neighborhood created by incoming edges is denoted
    by~$\glssymbol{inneighbor}(\vertexb)$.
    % 
    },
    % 
    symbol={\ensuremath{ \inneighbor }}
}

\newglossaryentry{outneighbor}
{
    name={\ensuremath{ \outneighbor }},
    sort={outneighbor},
    parent={neighbor},
    type={model},
    % 
    description={The neighborhood created by outgoing edges is
    denoted by~$\glssymbol{outneighbor}(\vertexb)$.
    % 
    },
    % 
    symbol={\ensuremath{ \outneighbor }}
}

\newglossaryentry{pathset}
{
    name={\ensuremath{ \paths(\vertexa,\vertexc) }},
    sort={pathSet},
    type={model},
    % 
    description={The set of all paths between two vertices~\vertexa and~\vertexc
    with~$\vertexa,\vertexc\in\glssymbol{vertices}$ is denoted by~\gls{pathset}.
    % 
    },
    % 
    symbol={\ensuremath{ \paths }}
}

\newglossaryentry{path}
{
    name={\ensuremath{ \pathu(\vertexa,\vertexc) }},
    sort={path},
    type={model},
    parent={pathset},
    % 
    description={A path starting at~\vertexa and ending at~\vertexc
    with~$\vertexa,\vertexc\in\glssymbol{vertices}$ is denoted by~\gls{path}.
    % 
    },
    % 
    symbol={\ensuremath{ \pathu }}
}

\newglossaryentry{dtppath}
{
    name={\ensuremath{ \fpath{\dtp}{\vertexa}{\vertexc} }},
    sort={dtpPath},
    type={model},
    parent={pathset},
    % 
    description={A~\dtp from~\vertexa to~\vertexb
    with~$\vertexa,\vertexc\in\glssymbol{vertices}$ is denoted by~\gls{dtppath}.
    % 
    },
    % 
    symbol={\ensuremath{ \fpath{\dtp}{\vertexa}{\vertexc} }}
}

\newglossaryentry{labels}
{
    name={\ensuremath{ \labels }},
    sort={labels},
    type={model},
    % 
    description={The set of nondominated labels at a
    vertex~$\vertexa\in\glssymbol{vertices}$ is denoted
    by~$\glssymbol{labels}$.
    % 
    },
    % 
    symbol={\ensuremath{ \labels }}
}

\newglossaryentry{paretolabels}
{
    name={\ensuremath{ \paretolabels(\vertexa) }},
    sort={paretoLabels},
    type={model},
    parent={labels},
    % 
    description={The Pareto set at vertex~$\vertexa\in\glssymbol{vertices}$.
    % 
    },
    % 
    symbol={\ensuremath{ \paretolabels }}
}

\newglossaryentry{label}
{
    name={\ensuremath{ \lab }},
    sort={label},
    type={model},
    parent={labels},
    % 
    description={A label~\glssymbol{label} includes the susceptance
    norm~$\bnorm{\glssymbol{path}}$ and the minimum
    capacity~$\glssymbol{mincapacity}(\glssymbol{path})$ of a path~$
    \glssymbol{path}$.
    % 
    },
    % 
    symbol={\ensuremath{ \lab }}
}

\newglossaryentry{priorityqueue}
{
    name={\ensuremath{ Q }},
    sort={priorityQueue},
    type={model},
    % 
    description={The priority queue is denoted by~\glssymbol{priorityqueue}.
    % 
    },
    % 
    symbol={\ensuremath{ Q }}
}

\newglossaryentry{factsbus}
{
    name={\ensuremath{ \factsbus }},
    sort={factsbus},
    type={model},
    % 
    description={The set of facts buses.
    % 
    A \emph{vertex hitting set} of~$\glssymbol{graph} =
    (\glssymbol{vertices},\glssymbol{edges})$ with respect to a class of
    graphs~$\mathcal G$ is a set of
    vertices~$\glssymbol{factsbus}\subseteq\glssymbol{vertices}$ such
    that~$\glssymbol{graph}-\glssymbol{factsbus} \in \mathcal G$.
    % 
    We call a subset of vertices~$\glssymbol{factsbus}^c$ $c$-pumpkin hitting
    set if there is a vertex
    subset~$\glssymbol{factsbus}^c\subseteq\glssymbol{vertices}(\glssymbol{graph})$
    such that~$
    \glssymbol{graph}-\glssymbol{factsbus}^c$ consists of no~$c$-pumpkin minor.
    % 
    },
    % 
    symbol={\ensuremath{ \factsbus }}
}

\newglossaryentry{factsbranch}
{
    name={\ensuremath{ \factsbranch }},
    sort={factsbranch},
    type={model},
    % 
    description={The set of facts buses.
    % 
    },
    % 
    symbol={\ensuremath{ \factsbranch }}
}

\newglossaryentry{dtpcentrality}
{
    name={\ensuremath{ c_{\sbc} }},
    sort={centralityBetweennessDtp},
    type={model},
    % 
    description={The~\dtp betweenness centrality is denoted by~$\glssymbol{dtpcentrality}$.
    % 
    },
    % 
    symbol={\ensuremath{ c_{\sbc} }}
}

\newglossaryentry{normalizedconstant}
{
    name={\ensuremath{ m_B }},
    sort={constantNormalizing},
    type={model},
    % 
    description={The normalizing constant for the betweenness centrality is
    denoted by~\glssymbol{normalizedconstant}.
    % 
    },
    % 
    symbol={\ensuremath{ m_B }}
}

\newglossaryentry{adjacencyMatrix}
{
    name={\ensuremath{ \adjacencyMatrix }},
    sort={matrixAdjacency},
    type={model},
    % 
    description={The \emph{oriented adjacency
    matrix}~$\glssymbol{adjacencyMatrix}\in\{-1, 0,
    1\}^{\fmagnitude{\glssymbol{vertices}}\times\fmagnitude{
    \glssymbol{vertices}}}$ represents the connections of a graph by vertex
    adjacencies meaning an entry in row~\vertexa and column~\vertexb is~$1$
    (respectively~$-1$) if there is an edge~$(\vertexa,\vertexb)\in
    \glssymbol{edges}(\glssymbol{graph})$ (respectively~$
    (\vertexb,\vertexa)\in\glssymbol{edges}(\glssymbol{graph})$). The entry
    is~$0$, if there is no such edge in the graph.
    % 
    }, 
    % 
    symbol={\ensuremath{\adjacencyMatrix }} 
    % 
}

\newglossaryentry{incidenceMatrix}
{
    name={\ensuremath{ \incidenceMatrix }},
    sort={matrixIncidence},
    type={model},
    % 
    description={
    The \emph{oriented incidence matrix}~$\incidenceMatrix\in\{-1,0,1\}^{
    \fmagnitude{\glssymbol{vertices}}\times\fmagnitude{\glssymbol{edges}}}$
    represents connections of a graph~\glssymbol{graph}. An entry in
    row~\vertexa and column~$(\vertexa,\vertexc)$ of the oriented incidence
    matrix~\glssymbol{incidenceMatrix} is~$1$ (respectively~$-1$) if there is
    an outgoing edge~$(\vertexa,\vertexc)$ (respectively incoming
    edge~$(\vertexc,\vertexa)$) at vertex~$\vertexa$, and~$0$ otherwise. An
    undirected version of the incidence matrix is defined
    by~$\incidenceMatrix\in\{0,1\}^{
    \fmagnitude{\glssymbol{vertices}}\times\fmagnitude{\glssymbol{edges}}}$,
    where an entry in row~\vertexa is~$1$ if there is an edge~$
    \{\vertexa,\vertexc\}\in\glssymbol{undirectededges}$.
    % 
    },
    % 
    symbol={\ensuremath{ \incidenceMatrix }} 
    % 
    }

\newglossaryentry{cycleMatrix}
{
    name={\ensuremath{ \cycleMatrix }},
    sort={matrixCircuit},
    type={model},
    % 
    description={
        The \emph{oriented circuit matrix}~$\glssymbol{cycleMatrix}\in
        \{-1,0,1\}^{
        \fmagnitude{\glssymbol{cycles}}\times \fmagnitude{\glssymbol{edges}}}$,
        where~\glssymbol{cycles} is the set of cycles. Assume we define a
        direction for each cycle. Thus, the matrix has a~$1$ entry if the edge
        is in the cycle and aligned with the cycle direction, $-1$ if it is
        opposite to the defined direction, and~$0$ otherwise.
        % 
    },
    symbol={\ensuremath{ \cycleMatrix }}
}

\newglossaryentry{cycles}
{
    name={\ensuremath{ \cycles }},
    sort={cycles},
    type={model},
    % 
    description={
        The set~\glssymbol{cycles} is called the set of cycles for a
        graph~\glssymbol{graph}.
    % 
    },
    symbol={\ensuremath{ \cycles }}
}

\newglossaryentry{cycle}
{
    name={\ensuremath{ \cycle }},
    sort={cycle},
    parent={cycles},
    type={model},
    % 
    description={
        A \emph{cycle} is a
        path~$\fpath{}{\source}{\sink}\in\paths(\source,\sink)$, where the first and the
        last vertex are identical meaning~$\source = \sink$. A cycle is called
        \emph{simple} if no vertex is visited twice with the exception of~$\source$
        and~$\sink$.
        % 
    },
    symbol={\ensuremath{ \cycle }}
}

\newglossaryentry{cutsetMatrix}
{
    name={\ensuremath{ \cutsetMatrix }},
    sort={matrixCutset},
    type={model},
    % 
    description={
        The \emph{oriented cut-set matrix}~$\cutsetMatrix\in\{-1, 0, 1\}^{
        \fmagnitude{\cutsets}\times\fmagnitude{\glssymbol{edges}}}$
        represents the graph~\glssymbol{graph} in terms of~$\cutsets$
        and~\glssymbol{edges}. The entry is~$1$ (respectively~$-1$) if the edge
        is in the cut-set and oriented in the predefined direction (respectively
        in the opposite direction), otherwise it is~$0$.
        % 
    },
    symbol={\ensuremath{ \cutsetMatrix }}
}