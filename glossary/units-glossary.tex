%%%%%%%%%%%%%%%%%%%%%%%%%%%%%%%%%%%%%%%%%%%%%%%%%%%%%%%%%%%%%%%%%%%%%%%%%%%%%%%
% Units
%%%%%%%%%%%%%%%%%%%%%%%%%%%%%%%%%%%%%%%%%%%%%%%%%%%%%%%%%%%%%%%%%%%%%%%%%%%%%%%

\newglossaryentry{pi}% Pi
{
    name={\ensuremath{\pi}},
    longform={pi},
    capitalform={Pi},
    symbol={\ensuremath{\pi}},
    sort={pi},
    type={unit},
    description={The ratio of a circle's circumference to its diameter.}
}

\newglossaryentry{twopi}% Pi
{
    name={\ensuremath{2\cdot\pi}},
    longform={two pi},
    capitalform={Two Pi},
    symbol={\ensuremath{2\pi}},
    parent={pi},
    sort={twopi},
    type={unit},
    description={A full period~$T$ that corresponds to a full cycle rotation of
    a vector in the Argand diagram that corresponds to~$\glssymbol{twopi}$ 
    (see~\cref{ch:foundations:fig:AC-voltage-current-angle-difference}).}
}

\newglossaryentry{ampere}% A
{
    name={\protect\si{\protect\ampere}},
    longform={ampere},
    capitalform={Ampere},
    sort={ampere},
    type={unit},
    % 
    description={The unit of electrical current is called~\emph{Ampere} (in
    short~\glssymbol{ampere}). It is one of the seven~\gls{si} base units and
    describes how much coulomb of current flows through a point per second
    (\si{\coulomb\per\second}).
    % 
    },
    % 
    symbol={\protect\si{\protect\coulomb}}
}

\newglossaryentry{coulomb}% C
{
    name={\protect\si{\protect\coulomb}},
    longform={coulomb},
    capitalform={Coulomb},
    sort={C},
    type={unit},
    % 
    description={The~\emph{Coulomb} corresponds to~$6.24\times 10^{18}$
    electrons and was introduced since the amount of one electron is simply to
    small. It is equal to the unit~\protect\si{\protect\ampere\second} and
    represents a~\gls{si} derived unit of electric charge. },
    % 
    symbol={\protect\si{\protect\coulomb}}
}

\newglossaryentry{joule}% J
{
    name={\protect\si{\protect\joule}},
    longform={joule},
    capitalform={Joule},
    sort={joule},
    type={unit},
    % 
    description={The unit of energy and thus, work is measured in~\emph{Joule}.
    From an electrical point of view it represents the electrical current
    density. It is a~\gls{si} derived unit measuring the (electrical) energy or
    work that is done~\ensuremath{
    % 
    \si[per-mode=fraction,fraction-function=\sfrac]{\newton\metre} 
    % 
    =\si[per-mode=fraction,fraction-function=\sfrac]{\volt\ampere\second} 
    % 
    =\si[per-mode=fraction,fraction-function=\sfrac]{\watt\second} 
    % 
    = \si[per-mode=fraction,fraction-function=\sfrac]
    {\sfrac{\kilogram~\metre\squared}{\second\squared}}
    % 
    }.
    % 
    },
    symbol={\protect\si{\protect\joule}}
}

\newglossaryentry{ohm}% Ohm
{
    name={\protect\si{\protect\ohm}},
    symbol={\si{\ohm}},
    sort={ohm},
    type={unit},
    % 
    description={The unit of electrical impedance~\impedance,
    resistance~\resistance, and reactance~\reactance is called~\emph{Ohm}. It is
    a derived~\gls{si} unit
    % 
    from~\protect\ensuremath{
    % 
    \protect\si{\volt\per\ampere} 
    % 
    = \si{\sfrac{\kilogram~\metre\squared}{\ampere\squared\second\cubed}}
    % 
    }.
    % 
    },
    % 
    symbol={\protect\si{\protect\ohm}}
}

\newglossaryentry{ohmmetre}% Om
{
    name={\protect\si{\protect\ohm\protect\metre}},
    longform={ohm metre},
    capitalform={Ohm Metre},
    sort={ohmmetre},
    type={unit},
    parent={ohm},
    % 
    description={The unit of resistivity~\resistivity is~\si{\ohm\metre}. It
    represents the property how well a material resists electric current. It is
    reciprocal to the electrical conductivity~\conductivity and is a
    derived~\gls{si} unit from~\ensuremath{
    % 
    \si{\sfrac{\kilogram\metre\cubed}{\ampere\squared\second\cubed}}
    % 
    }. Note that another definition is~\ensuremath{\resistivity:=\resistance
    \frac{A}{\ell}}, where~$A$ is the area (wired gauge) and~$\ell$ the length
    of the material.
    % 
    },
    % 1m \times 1m \times 1m solid cube contacts on oposite sides -|_|-
    % resistance between these contacts 1 ohm -> 1 ohm metre
    % 
    % intrinsic property -> always the same for same volume
    % resistance differs on length to area ratio, i.e., longer length means more
    % resistance, than higher gauge and smaller length
    symbol={\protect\si{\protect\sfrac{\protect\siemens}{\protect\metre}}}
}

\newglossaryentry{si}% SI
{
    name={SI},
    longform={international system of units},
    capitalform={International System of Units},
    sort={si},
    type={unit},
    % 
    description={The International System of Units (\gls{si}) is a metric system
    that helps to prevent conversion problems as it specify the seven base units
    that are ampere~\protect\si{\protect\ampere}, candela~\protect
    \si{\protect\candela}, kelvin~\protect\si{\protect\kelvin},
    kilogram~\protect\si{\protect\kilogram}, metre~\protect\si{\protect\metre},
    mole~\protect\si{\protect\mole}, and seconds~\protect\si{\protect\second}. },
    % 
    symbol={SI}
}

\newglossaryentry{siemens}% S
{
    name={\protect\si{\protect\siemens}},
    longform={siemens},
    capitalform={Siemens},
    sort={siemens},
    type={unit},
    % 
    description={The unit of electrical admittance~\admittance,
    conductance~\conductance, and susceptance~\susceptance is
    Siemens~\si{\siemens}. It is a derived~\gls{si} unit
    % 
    from~\ensuremath{
    \si{\sfrac{1}{\ohm}}
    % 
    = \si{\ampere\per\volt}
    % 
    = \si{\sfrac{\ampere\squared\second\cubed}{\kilogram~\metre\squared} }
    }.
    %
    It also shows that Siemens is reciprocal to Ohm.
    % increase of one volt increases voltages by one if 1 S
    },
    % 
    symbol={\protect\si{\protect\siemens}}
}

\newglossaryentry{siemenspermetre}% Wh
{
    name={\protect\si{\protect\sfrac{\protect\siemens}{\protect\metre}}},
    longform={siemens per metre},
    capitalform={Siemens per Metre},
    sort={siemenspermetre},
    type={unit},
    parent={siemens},
    % 
    description={The unit of conductivity~\conductivity is~\si
    [per-mode=fraction,fraction-function=\sfrac]{\siemens\per\metre}. It
    represents the property how well a material (conductor) conducts electric
    current. It is reciprocal to the electrical resistivity~\resistivity and is
    a derived SI unit from~\ensuremath{
    % 
    \si{\sfrac{\ampere\squared\second\cubed}{\kilogram~\metre\cubed}}
    % 
    }.
    % 
    },
    % 
    symbol={\protect\si{\protect\sfrac{\protect\siemens}{\protect\metre}}}
}

\newglossaryentry{volt}% V
{
    name={\protect\si{\protect\volt}},
    longform={volt},
    capitalform={Volt},
    sort={volt},
    type={unit},
    % 
    description={The unit of voltage~\voltage is called~\emph{Volt} representing
    a potential difference and a electromotive force in electrical circuits. It
    is a~\gls{si} derived unit measuring either the work that is done on one
    ampere or an alternative formulation the force that take effect on an
    electrical
    % 
    charge~\ensuremath{
    % 
    \si[per-mode=fraction,fraction-function=\sfrac]{\watt\per\ampere}
    % 
    = \si[per-mode=fraction,fraction-function=\sfrac]{\joule\per\coulomb} 
    % 
    =
    \si[per-mode=fraction,fraction-function=\sfrac]{{\newton\metre}\per{\ampere\second}}}.
    },
    % 
    symbol={\protect\si{\protect\volt}}
}

\newglossaryentry{watt}% W
{
    name={\protect\si{\protect\watt}},
    longform={watt},
    capitalform={Watt},
    sort={watt},
    type={unit},
    % 
    description={The unit of power such as real power~\realpower is measured
    in~\emph{Watt}. It is a~\gls{si} derived unit measuring the (electrical)
    energy conversion per second (representing a degree of efficiency) that is
    done~\ensuremath{
    % 
    \si[per-mode=fraction,fraction-function=\sfrac]{\joule\per\second} 
    % 
    = \si[per-mode=fraction,fraction-function=\sfrac]
    {\sfrac{\kilogram~\metre\squared}{\second\cubed}}
    % 
    }. Note that it is used for the real power. For the complex power and
    reactive power the units~\si{\volt\ampere} and~\si{\var} are used,
    respectively.
    % 
    },
    symbol={\protect\si{\protect\watt}}
}

\newglossaryentry{voltampere}% VA
{
    name={\protect\si{\protect\volt\protect\ampere}},
    longform={volt ampere},
    capitalform={Volt Ampere},
    sort={va},
    type={unit},
    % 
    description={ The unit of the complex and apparent power is Volt
    Ampere~\glssymbol{voltampere}. For more information
    see~\glscapitalform{watt}~\glssymbol{watt}.
    % 
    },
    % 
    symbol={\protect\si{\protect\volt\protect\ampere}}
}

\newglossaryentry{var}% VAr
{
    name={\protect\si{\protect\var}},
    longform={volt ampere reactive},
    capitalform={Volt Ampere Reactive},
    sort={var},
    type={unit},
    % 
    description={ The unit of the reactive power is Volt
    Ampere reactive~\glssymbol{var}. For more information
    see~\glscapitalform{watt}~\glssymbol{watt}.
    % 
    },
    % 
    symbol={\protect\si{\protect\var}}
}

\newglossaryentry{kv}% kV
{
    name={\protect\si{\protect\kV}},
    longform={kilovolt},
    capitalform={Kilovolt},
    sort={kv},
    parent={watt},
    type={unit},
    % 
    description={One~\si{\kV} corresponds to $1\,000$~\si{\volt}.},
    % 
    symbol={\protect\si{\protect\kV}}
}

\newglossaryentry{kw}% kW
{
    name={\protect\si{\protect\kW}},
    longform={kilowatt},
    capitalform={Kilowatt},
    sort={kw},
    parent={watt},
    type={unit},
    % 
    description={One~\si{\kW} corresponds to $1\,000$~\si{\watt}.},
    % 
    symbol={\protect\si{\protect\kW}}
}

\newglossaryentry{kwh}% kWh
{
    name={\protect\si{\protect\kWh}},
    longform={kilowatt hour},
    capitalform={Kilowatt Hour},
    sort={kwh},
    parent={watt},
    type={unit},
    % 
    description={One~\si{\kWh} corresponds to~$3.6$~\si{\mega\joule}. It
    basically describes that one~\gls{kw} is in average used over one hour of
    time. Note that this is a non-\gls{si} unit and mainly used for electricity
    bills.},
    % 
    symbol={\protect\si{\protect\kWh}}
}

\newglossaryentry{mw}% MW
{
    name={\protect\si{\protect\MW}},
    longform={megawatt},
    capitalform={Megawatt},
    sort={mw},
    parent={watt},
    type={unit},
    description={One~\si{\MW} corresponds to $1\,000\,000$~\gls{watt}.},
    symbol={\protect\si{\protect\MW}}
}

\newglossaryentry{mwh}% mWh
{
    name={\protect\si{\protect\mega\protect\watt\protect\hour}},
    longform={megawatt hour},
    capitalform={Megawatt Hour},
    sort={mwh},
    parent={watt},
    type={unit},
    description={One~\si{\mega\watt} corresponds to $10^{6}$~\si{\watt}.},
    symbol={\protect\si{\protect\mega\protect\watt\protect\hour}}
}

\newglossaryentry{miwh}% mWh
{
    name={\protect\si{\protect\milli\protect\watt\protect\hour}},
    longform={milliwatt hour},
    capitalform={Milliwatt Hour},
    sort={miwh},
    parent={watt},
    type={unit},
    description={One~\si{\milli\watt} corresponds to $10^{-3}$~\si{\watt} and
    thus, a~\si{\protect\milli\protect\watt\protect\hour} corresponds to
    3.6~\gls{joule}.},
    symbol={\protect\si{\protect\milli\protect\watt\protect\hour}}
}

\newglossaryentry{hertz}%
{
    name={\protect\si{\protect\hertz}},
    longform={hertz},
    capitalform={Hertz},
    sort={hertz},
    type={unit},
    description={One~\si{\hertz} corresponds to one~$\si{\second}^{-1}$. It
    is used in terms of frequencies.},
    symbol={\protect\si{\protect\hertz}}
}

% \newglossaryentry{boltzmannKonstante}% mWh
% {
%     name={\ensuremath{k_B}},
%     longform={Boltzmann-Constant},
%     capitalform={Boltzmann-Constant},
%     sort={boltzmann},
%     type={unit},
%     description={The boltzmann constant is a exact physical constant
%     with~$k_B = 1.380\, 649\cdot 10^{-23}\si{\joule\per\kelvin}$ and comes from
%     the statistical mechanic.
%     }, 
%     symbol={\ensuremath{k_B}}
% }
% 


% Units
\glsadd{ampere}
\glsadd{coulomb}
\glsadd{hertz}
\glsadd{joule}
\glsadd{kv}
\glsadd{kw}
\glsadd{kwh}
\glsadd{miwh}
\glsadd{mw}
\glsadd{mwh}
\glsadd{ohm}
\glsadd{ohmmetre}
\glsadd{pi}
\glsadd{si}
\glsadd{siemens}
\glsadd{siemenspermetre}
\glsadd{twopi}
\glsadd{var}
\glsadd{volt}
\glsadd{voltampere}
\glsadd{watt}