\newglossaryentry{susceptance}
{
    name={\susceptance},
    sort={function susceptance},
    type={model},
    parent={network},
    % 
    description={The susceptance is a function~$
    \glssymbol{susceptance} 
    \colon
    \glssymbol{undirectededges} 
    \to 
    \posreals$ 
    defined by~$
    \glssymbol{susceptance}(\vertexa,\vertexc)
    \coloneqq 
    -
    \nicefrac{
        \glssymbol{reactance}(\vertexa,\vertexc)
    }{
        (\glssymbol{resistance}(\vertexa,\vertexc)^2
        + 
        \glssymbol{reactance}(\vertexa,\vertexc)^2)
    }$ for all~$\{\vertexa,\vertexc\}\in\glssymbol{undirectededges}$. Note that
    for~\gls{dc} models it is simply the reciprocal of the
    reactance~\glssymbol{reactance}. Roughly speaking it represents how easy an
    electron is able to pass a material.
    % 
    },
    % 
    symbol={\susceptance}
}

\newglossaryentry{susceptancenorm}
{
    name={\ensuremath{ \bnorm{\fpath{}{\vertexa}{\vertexb}} } },
    sort={susceptance norm},
    type={model},
    % 
    description={The susceptance norm of a path~$
    \fpath{}{\vertexa}{\vertexb}
    $ is defined by~$
    \bnorm{\fpath{}{\vertexa}{\vertexb}}
    \coloneqq
    \sum_{\edge\in\fpath{}{\vertexa}{\vertexb}}
    \glssymbol{susceptance}(\edge)^{-1}
    $. The susceptance norm is a norm, since it fulfills the axioms given
    by~\textcite{Ban22}.
    % 
    },
    % 
    symbol={\ensuremath{ \bnorm{\fpath{}{\vertexa}{\vertexb}} } }
}

\newglossaryentry{conductance}
{
    name={\conductance},
    sort={function conductance},
    type={model},
    parent={network},
    % 
    description={The conductance~\glssymbol{conductance} is a
    function~$
    \glssymbol{conductance} 
    \colon
    \glssymbol{undirectededges} 
    \to 
    \reals
    $ defined by~$
    \glssymbol{conductance}(\vertexa,\vertexc) 
    \coloneqq 
    \nicefrac{
        \glssymbol{resistance}(\vertexa,\vertexc)
    }{  
        (\glssymbol{resistance}(\vertexa,\vertexc)^2 
        + 
        \glssymbol{reactance}(\vertexa,\vertexc)^2)
    }$ for all~$\{\vertexa,\vertexc\}\in\glssymbol{undirectededges}$. Note that
    for~\gls{dc} models it is simply the reciprocal of the
    resistance~\glssymbol{resistance}. Roughly speaking it represents how easy
    an electron is able to pass a material.
    % 
    },
    % 
    symbol={\conductance}
}

\newglossaryentry{resistance}
{
    name={\resistance},
    sort={function resistance},
    type={model},
    parent={network},
    % 
    description={The resistance is a
    function~$\glssymbol{resistance}\colon
    \glssymbol{undirectededges}\to\posreals\cup\{\infty\}$ representing the real
    part of the impedance. Note that for~\gls{dc} models it is simply the
    reciprocal of the~\glssymbol{conductance}. Roughly speaking it represents
    how easy an electron is able to pass a material.
    % 
    },
    % 
    symbol={\resistance}
}

\newglossaryentry{impedance}
{
    name={\impedance},
    sort={function impedance},
    type={model},
    parent={network},
    % 
    description={The impedance is a
    function~$\glssymbol{impedance}\colon\glssymbol{undirectededges}\to\complexes$
    defined by~$
        \glssymbol{impedance}(\vertexa,\vertexc) \coloneqq 
        \frac{1}{\glssymbol{admittance}(\vertexa,\vertexc)} 
    =
        \glssymbol{resistance}(\vertexa,\vertexc) 
    + 
        \imgPart\cdot\glssymbol{reactance}(\vertexa,\vertexc)
    $
    for all~$\{\vertexa,\vertexc\}\in\glssymbol{undirectededges}$.
    % 
    },
    % 
    symbol={\impedance}
}

\newglossaryentry{reactance}
{
    name={\reactance},
    sort={function reactance},
    type={model},
    parent={network},
    % 
    description={The reactance is a
    function~$\glssymbol{reactance}\colon
    \glssymbol{undirectededges}\to\posreals\cup\{\infty\}$ representing the
    imaginary part of the impedance. Note that for~\gls{dc} models it is simply
    the reciprocal of the~\glssymbol{susceptance}. Roughly speaking it
    represents how easy an electron is able to pass a material.
    % 
    },
    % 
    symbol={\reactance}
}

\newglossaryentry{admittance}
{
    name={\admittance},
    sort={function admittance},
    type={model},
    parent={network},
    % 
    description={
    % 
    The admittance is a
    function~$\glssymbol{admittance}\colon\glssymbol{undirectededges}\to\complexes$~(\cref{ch:foundations:sec:power-flow-analyses:eq:admittance}), where
    $\glssymbol{admittance}(\vertexa,\vertexb)$~(\cref{ch:foundations:sec:power-flow-analyses:eq:admittance}) is defined by
    the
    conductance~$\glssymbol{conductance}(\vertexa,\vertexb)$~(\cref{ch:foundations:sec:power-flow-analyses:eq:conductance})
    and the
    susceptance~$\glssymbol{susceptance}(\vertexa,\vertexb)$~(\cref{ch:foundations:sec:power-flow-analyses:eq:susceptance})
    that define how easy the current is able to flow through an element such as
    a transmission line~$\{\vertexa,\vertexb\}\in\glssymbol{undirectededges}$.
    % 
    },
    % 
    symbol={\admittance}
}

\newglossaryentry{complexpower}
{
    name={\complexpower},
    sort={complex power},
    type={model},
    % parent={network},
    % 
    description={The complex power
    function~$\complexpower\colon\vertices\to\complexes$ is the sum of the real
    and reactive power. For time varying models, the function is called
    instantaneous electric power, which is a
    function~$\glssymbol{complexpower}\colon\glssymbol{vertices}\times\reals\to\complexes$
    representing the instantaneous power at
    vertex~$\vertexa\in\glssymbol{vertices}$ for
    timestamp~$\timestamp\in\reals$. It is usually denoted by~$p$, but the
    additional parameter separates the constant
    term~$\glssymbol{complexpower}(\vertexa)$ clearly from the time varying
    term~$\glssymbol{complexpower}(\vertexa,\timestamp)$.
    % 
    },
    % 
    symbol={\complexpower}
}

\newglossaryentry{complexpowermax}
{
    name={\complexpowermax},
    sort={complex power maximum},
    type={model},
    % 
    description={The apparent power maximum~\glssymbol{complexpowermax} is a
    total thermal line limitation in terms of power. },
    % 
    symbol={\complexpowermax}
}

\newglossaryentry{currentmax}
{
    name={\currentmax},
    sort={current maximum},
    type={model},
    % 
    description={The current power maximum~\glssymbol{currentmax} is a total
    thermal line limitation in terms of current~\glssymbol{current}. },
    % 
    symbol={\currentmax}
}

\newglossaryentry{realpower}
{
    name={\realpower},
    sort={real power},
    type={model},
    % 
    description={The real power~$\glssymbol{realpower}$ is also called active power and
    represents the real part of the complex power~\glssymbol{complexpower}.
    % 
    },
    % 
    symbol={\realpower}
}

\newglossaryentry{realpowermax}
{
    name={\realpowermax},
    sort={real power maximum},
    type={model},
    % 
    description={The real power maximum~\glssymbol{realpowermax} is the real
    part of the thermal line limitation in terms of power.
    % 
    }, 
    % 
    symbol={\realpowermax}
}

\newglossaryentry{realpowermin}
{
    name={\realpowermin},
    sort={real power maximum},
    type={model},
    % 
    description={The real power minimum~\glssymbol{realpowermin} is the real
    part of the thermal line limitation in terms of power.
    % 
    }, 
    % 
    symbol={\realpowermin}
}

\newglossaryentry{realpowergeneration}
{
    name={\realpowergeneration},
    sort={real power generation},
    type={model},
    parent={network},
    % 
    description={The generators' real power output at vertex~$\vertexa \in
    \glssymbol{generators}$.},
    % 
    symbol={\realpowergeneration}
}

\newglossaryentry{realpowergenerationmax}
{
    name={\realpowergenerationmax},
    sort={real power generation maximum},
    type={model},
    parent={network},
    % 
    description={The generators' real power upper bound at vertex~$\vertexa \in
    \glssymbol{generators}$.
    % 
    },
    % 
    symbol={\realpowergenerationmax}
}

\newglossaryentry{realpowergenerationmin}
{
    name={\realpowergenerationmin},
    sort={real power generation minimum},
    type={model},
    parent={network},
    % 
    description={The generators' real power lower bound at vertex~$\vertexa \in
    \glssymbol{generators}$.
    % 
    },
    % 
    symbol={\realpowergenerationmin}
}

\newglossaryentry{realpowerdemand}
{
    name={\realpowerdemand},
    sort={real power demand},
    type={model},
    % parent={network},
    % 
    description={The demands' real power at vertex~$\vertexa \in
    \glssymbol{consumers}$.
    % 
    },
    % 
    symbol={\realpowerdemand}
}

\newglossaryentry{realpowerdemandmax}
{
    name={\realpowerdemandmax},
    sort={real power demand maximum},
    type={model},
    parent={network},
    % 
    description={The demands' real power upper bound at vertex~$\vertexa \in
    \glssymbol{consumers}$.
    % 
    },
    % 
    symbol={\realpowerdemandmax}
}

\newglossaryentry{realpowerdemandmin}
{
    name={\realpowerdemandmin},
    sort={real power demand minimum},
    type={model},
    parent={network},
    % 
    description={The demands' lower real power bound at vertex~$\vertexa \in
    \glssymbol{consumers}$.
    % 
    },
    % 
    symbol={\realpowerdemandmin}
}

\newglossaryentry{reactivepower}
{
    name={\reactivepower},
    sort={reactive power},
    type={model},
    % 
    description={The reactive power~\glssymbol{reactivepower} is also called
    phantom power and represents the imaginary part of the complex
    power~\glssymbol{complexpower}.
    % 
    },
    % 
    symbol={\reactivepower}
}

\newglossaryentry{reactivepowermax}
{
    name={\reactivepowermax},
    sort={reactive power maximum},
    type={model},
    parent={network},
    % 
    description={The reactive power maximum~\glssymbol{reactivepowermax} is the
    imaginary part of the thermal line limitation.
    % 
    }, 
    % 
    symbol={\reactivepowermax}
}

\newglossaryentry{reactivepowermin}
{
    name={\reactivepowermin},
    sort={reactive power minimum},
    type={model},
    parent={network},
    % 
    description={The reactive power minimum~\glssymbol{reactivepowermin} is the
    imaginary part of the thermal line limitation.
    % 
    }, 
    % 
    symbol={\reactivepowermin}
}

\newglossaryentry{reactivepowergeneration}
{
    name={\reactivepowergeneration},
    sort={reactive power generation},
    type={model},
    % 
    description={The generators' reactive power output at vertex~$\vertexa \in
    \glssymbol{generators}$.
    % 
    },
    % 
    symbol={\reactivepowergeneration}
}

\newglossaryentry{reactivepowergenerationmax}
{
    name={\reactivepowergenerationmax},
    sort={reactive power generation maximum},
    type={model},
    parent={network},
    % 
    description={The generators' reactive power upper bound at vertex~$\vertexa
    \in \glssymbol{generators}$.
    % 
    },
    % 
    symbol={\reactivepowergenerationmax}
}

\newglossaryentry{reactivepowergenerationmin}
{
    name={\reactivepowergenerationmin},
    sort={reactive power generation minimum},
    type={model},
    parent={network},
    % 
    description={The generators' reactive power lower bound at vertex~$\vertexa
    \in \glssymbol{generators}$.
    % 
    },
    % 
    symbol={\reactivepowergenerationmin}
}

\newglossaryentry{reactivepowerdemand}
{
    name={\reactivepowerdemand},
    sort={reactive power demand},
    type={model},
    % 
    description={The demands' reactive power demand at a consumer
    vertex~$\vertexa \in \glssymbol{consumers}$.
    % 
    },
    % 
    symbol={\reactivepowerdemand}
}

\newglossaryentry{reactivepowerdemandmax}
{
    name={\reactivepowerdemandmax},
    sort={reactive power demand maximum},
    type={model},
    parent={network},
    % 
    description={The demands' reactive power upper bound at vertex~$\vertexa \in
    \glssymbol{consumers}$.
    % 
    },
    % 
    symbol={\reactivepowerdemandmax}
}

\newglossaryentry{reactivepowerdemandmin}
{
    name={\reactivepowerdemandmin},
    sort={reactive power demand minimum},
    type={model},
    parent={network},
    % 
    description={The demands' reactive power lower bound at vertex~$\vertexa \in
    \glssymbol{consumers}$.
    % 
    },
    % 
    symbol={\reactivepowerdemandmin}
}

\newglossaryentry{current}
{
    name={\current},
    sort={current},
    type={model},
    % 
    description={ The current is a
    function~$\glssymbol{current}\colon\glssymbol{vertices}\to\reals$ that
    represents the electrons that move through an element per second an thus, is
    measured in Ampere~\gls{ampere}. The time varying function (in a dynamic
    network setting) of current is defined
    by~$\glssymbol{current}\colon\glssymbol{vertices}\times\reals\to\reals$ that
    represents the instantaneous
    current~$\glssymbol{current}(\vertexa,\glssymbol{timestamp})$ at
    vertex~$\vertexa\in\glssymbol{vertices}$ for
    timestamp~$\glssymbol{timestamp}\in\reals$.
    % 
    },
    %
    symbol={\current}
}

\newglossaryentry{currentrms}
{
    name={\currentrms},
    sort={currentrms},
    type={model},
    % 
    description={The~\acrlong{rms} (\gls{rms}) value of a current
    magnitude~$\fmagnitude{\glssymbol{current}(\vertexa)}$ at
    vertex~$\vertexa\in\glssymbol{vertices}$ is defined
    by~$\nicefrac{\fmagnitude{\glssymbol{current}(\vertexa)}}{
    \sqrt{2}}$. It
    represents the effective value of a sinusoid current waveform. Note that
    the~\gls{rms} value is only used for time varying sinusoid functions and not
    for~\gls{dc} or~\gls{ac} models, where we assume time invariance and thus,
    the functions become constant over time.
    % 
    },
    %
    symbol={\currentrms}
}

\newglossaryentry{angularfrequency}
{
    name={\angularfrequency},
    sort={angularfrequency},
    type={model},
    % 
    description={The velocity of the angle (here voltage
    angle~\glssymbol{voltageangle} and current angle~\glssymbol{currentangle})
    rotation is defined by the angular
    frequency~$\glssymbol{angularfrequency}\coloneqq\nicefrac{d\aangle}{d\timestamp}$.
    Thus, it describes how the phase changes of a sinusoid function. In a
    time-invariant setting, we have a constant rotation velocity
    meaning~$\glssymbol{angularfrequency}\coloneqq\nicefrac{d\aangle} {d\timestamp} =
    \gls{twopi}\glssymbol{frequency}$. In the Argand diagram
    (see~\cref{ch:foundations:fig:AC-voltage-current-angle-difference}) the
    rotation speed of the voltage vector~\glssymbol{voltage} and current
    vector~\glssymbol{current} (counter-clock-wise) is meant.
    % 
    },
    % 
    symbol={\angularfrequency}
}

\newglossaryentry{frequency}
{
    name={\frequency},
    sort={frequency},
    type={model},
    % 
    description={The frequency is defined
    by~$\glssymbol{frequency}\coloneqq\frac{1}{T}$, where~$T$ is the time to
    complete one cycle of~$\glssymbol{twopi}$ radians.
    % 
    },
    % 
    symbol={\frequency}
}

\newglossaryentry{currentangle}
{
    name={\iangle},
    sort={current angle},
    type={model},
    % 
    description={The current angle is a
    function~$\glssymbol{currentangle}\colon\glssymbol{vertices}\to\reals$ at a
    vertex~$\vertexa\in\glssymbol{vertices}$ that represents an initial
    potential at a vertex~\vertexa. The current angle~\glssymbol{currentangle}
    is the initial angle between the current vector and the x-axis in the Argand
    diagram (see~\cref{ch:foundations:fig:AC-voltage-current-angle-difference}).
    % 
    },
    % 
    symbol={\iangle}
}

\newglossaryentry{voltage}
{
    name={\voltage},
    sort={voltage},
    type={model},
    % 
    description={The voltage is a
    function~$\glssymbol{voltage}\colon\glssymbol{vertices}\to\reals$ that
    represents the push of current~\glssymbol{current} and is measured in Volt.
    Note that we will talk some times of a voltage drop at an
    element~$(\vertexa,\vertexc)\in\glssymbol{edges}$. Then voltage is a
    function~$\glssymbol{voltage}\colon\glssymbol{edges}\to\reals$. If we use
    the vertex based definition, we just neglect the reference point, which is
    ground~$0$.
    % 
    },
    %
    symbol={\voltage}
}

\newglossaryentry{voltagemin}
{
    name={\voltagemin},
    sort={voltage minimum},
    type={model},
    % 
    description={The voltage magnitude's~$\fmagnitude{\glssymbol{voltage}
    (\vertexa)}$ lower bound is denoted by~\glssymbol{voltagemin}.
    % 
    },
    %
    symbol={\voltagemin}
}

\newglossaryentry{voltagemax}
{
    name={\voltagemax},
    sort={voltage maximum},
    type={model},
    % 
    description={The voltage magnitude's~$\fmagnitude{\glssymbol{voltage}
    (\vertexa)}$ upper bound is denoted by~\glssymbol{voltagemax}.
    % 
    },%
    symbol={\voltagemax}
}

\newglossaryentry{voltagerms}
{
    name={\voltagerms},
    sort={voltagerms},
    type={model},
    % 
    description={The~\acrlong{rms} (\gls{rms}) value of a voltage
    magnitude~$\fmagnitude{\glssymbol{voltage}(\vertexa)}$ is defined
    by~$\nicefrac{\fmagnitude{\glssymbol{voltage}(\vertexa)}}{\sqrt{2}}$ at
    vertex~$\vertexa\in\glssymbol{vertices}$. It represents the effective value
    of the voltage. Note that the~\gls{rms} value is only used for time varying
    sinusoid functions.
    % 
    },
    %
    symbol={\voltagerms}
}

\newglossaryentry{voltageangle}
{
    name={\vangle},
    sort={voltage angle},
    type={model},
    % 
    description={The voltage angle (also called phase angle or theta angle) is a
    function~$\glssymbol{voltageangle}\colon\glssymbol{vertices}\to\reals$ at a
    vertex~\vertexa that represents a potential at a vertex~\vertexa. The
    voltage angle is the angle between the voltage vector and the x-axis in the
    Argand diagram
    (see~\cref{ch:foundations:fig:AC-voltage-current-angle-difference}).
    % 
    },
    % 
    symbol={\vangle}
}

\newglossaryentry{minvoltageangle}
{
    name={\ensuremath{\vanglemin}},
    sort={voltage angle minimum},
    type={model},
    parent={voltageangle},
    % 
    description={The voltage angle's lower bound is denoted
    by~$\glssymbol{vangle}_{\min}(\vertexa)$ at
    vertex~$\vertex\in\glssymbol{vertices}$.
    % 
    },
    % 
    symbol={\ensuremath{\vanglemin}}
}

\newglossaryentry{maxvoltageangle}
{
    name={\ensuremath{\vanglemax}},
    sort={voltage angle maximum},
    type={model},
    parent={voltageangle},
    % 
    description={The voltage angle's upper bound is denoted
    by~$\glssymbol{vangle}_{\max}(\vertexa)$ at
    vertex~$\vertex\in\glssymbol{vertices}$.
    % 
    },
    % 
    symbol={\ensuremath{\vanglemax}}
}

\newglossaryentry{voltageangleshift}
{
    name={\ensuremath{\vangleshift}},
    sort={voltage angle shift},
    type={model},
    parent={voltageangle},
    % 
    description={An additional voltage angle
    shift~$\glssymbol{voltageangleshift}(\vertexa,\vertexb)$ at an edge~$
    \{\vertexa,\vertexb\}\in\glssymbol{undirectededges}$ influences the voltage
    angle difference such that~$\glssymbol{voltageangle}(\vertexc)-
    \glssymbol{voltageangle}(\vertexa)-\glssymbol{voltageangleshift}
    (\vertexa,\vertexb)$. A electrical devices that causes such a shift is a
    transformer (phase shifter). Thus, this angle is often called transformer
    (phase shifter) final angle.
    % 
    },
    % 
    symbol={\ensuremath{\vangleshift}}
}

\newglossaryentry{tapratio}
{
    name={\ensuremath{\tapratio}},
    sort={tap ratio},
    type={model},
    % 
    description={The tap ratio (also known as turn ratio) is defined by~$\glssymbol{tapratio}(\vertexa,\vertexc)\coloneqq\nicefrac{\fmagnitude{
    \glssymbol{voltage}(\vertexc)}} {\fmagnitude{\glssymbol{voltage}(\vertexa)}} =
    \nicefrac{\glssymbol{voltage}(\vertexc)}{\glssymbol{voltage} (\vertexa)} = -
    \nicefrac{\glssymbol{current}(\vertexa, \vertexc)}{\glssymbol{current} (\vertexc, \vertexa)}$.
    % 
    },
    % 
    symbol={\ensuremath{\tapratio}}
}


\newglossaryentry{linechargingadmittance}
{
    name={\ensuremath{\linechargingadmittance}},
    sort={tap ratio},
    type={model},
    % 
    description={The line charging
    admittance~$\glssymbol{linechargingadmittance}(\vertexa,\vertexc)$ is a
    parameter of a transmission line.},
    % 
    symbol={\ensuremath{\linechargingadmittance}}
}

\newglossaryentry{linechargingsusceptance}
{
    name={\ensuremath{\linechargingsusceptance}},
    sort={tap ratio},
    type={model},
    % 
    description={ The line charging
    susceptance~$\glssymbol{linechargingsusceptance}(\vertexa,\vertexc)$ is a
    parameter of a transmission
    line~$\{\vertexa,\vertexc\}\in\glssymbol{undirectededges}$ as medium to long
    transmission lines tend to have an inherent capacitance. 
    % 
    },
    % 
    symbol={\ensuremath{\linechargingsusceptance}}
}

\newglossaryentry{voltageangledifference}
{
    name={\ensuremath{ \deltaangle }},
    sort={voltage angle difference},
    type={model},
    parent={voltageangle},
    % 
    description={The voltage angle difference of a particular
    edge~$(\vertexa,\vertexc)\in\glssymbol{edges}$ is defined by~$
    \glssymbol{voltageangledifference}(\vertexa,\vertexc)\coloneqq 
    \glssymbol{voltageangle}(\vertexc) - \glssymbol{voltageangle}(\vertexa)$.
    For a~\vertexa-\vertexb-path~$\fpath{}{\vertexa}{\vertexc}$ it is denoted
    by~$\glssymbol{voltageangledifference}(\fpath{}{\vertexa}{\vertexc})$.
    % 
    },
    % 
    symbol={\ensuremath{ \deltaangle }}
}

\newglossaryentry{voltageangledifferencemin}
{
    name={\ensuremath{ \deltavanglemin }},
    sort={voltage angle difference minimum},
    type={model},
    parent={voltageangle},
    % 
    description={The voltage angle difference's lower
    bound~\glssymbol{voltageangledifferencemin} restricts the flow.
    % 
    },
    % 
    symbol={\ensuremath{ \deltavanglemin }}
}

\newglossaryentry{voltageangledifferencemax}
{
    name={\ensuremath{ \deltavanglemax }},
    sort={voltage angle difference maximum},
    type={model},
    parent={voltageangle},
    % 
    description={The voltage angle difference's upper
    bound~\glssymbol{voltageangledifferencemax} restricts the flow.
    % 
    },
    % 
    symbol={\ensuremath{ \deltavanglemax }}
}
